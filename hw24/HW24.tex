% This template retrieved from "https://guides.nyu.edu/LaTeX/templates"

\documentclass[11pt]{article} 

\usepackage{geometry} 
\usepackage{amsmath}  
\usepackage{graphicx} 
\usepackage{amssymb}
\usepackage{amscd}
\usepackage{amsfonts}
\usepackage[shortlabels]{enumitem}

\newcommand{\prob}[3]{\begin{flushleft}
        \textbf{Problem #1}: \\
        #2 
		\textbf{Solution:} 
		#3

\end{flushleft}}

\newcommand{\admit}{
  \begin{flushright}
    \textbf{Admitted}
  \end{flushright}
}

\newcommand{\lr}[1]{
  \langle #1 \rangle
}

\newcommand{\makeHWtitle}[1]{
    \begin{center}
    \Large{Homework #1 - MATH 791} 
        \vspace{5pt}
        
        \normalsize{Will Thomas}
        \vspace{5pt}
    \end{center}
}

\begin{document}

\makeHWtitle{24}

\prob{1}{
  Write out addition and multiplication tables for the field $\mathbb{Z}_2(\alpha)$ in problem 5 of the previous assignment. \\
}{\\
  This problem is a bit confusing, we are supposed to not need to
  any explicit terms for $\alpha$.
}

\prob{2}{
  Now let $p(x)$ and $\alpha$ be as in problems 1 and 2 from Homework 23.
  Determine whether or not $\mathbb{Z}_3(\alpha)$ the splitting field for $p(x)$
  over $\mathbb{Z}_3$. \\
}{\\
  \admit
}

\prob{3}{
Let $p,q \in \mathbb{Z}$ be distinct prime numbers. Show that $[\mathbb{Q}(\sqrt{p}, \sqrt{q}) : \mathbb{Q}] = 4$. \\
}{\\
We know that the minimal polynomial $f(x)$ such that $f(\sqrt{p}) = 0$ is $f(x) = x^2 - p$. Thus $[\mathbb{Q}(\sqrt{p}) : \mathbb{Q}] = 2$.
Additionally, we know that $g(x) \in \mathbb{Q}(\sqrt{p})[x]$ such that $g(\sqrt{q}) = 0$ is $g(x) = x^2 - q$. This is due to the primes being distinct, and this being a UFD.
Thus $[\mathbb{Q}(\sqrt{p})(\sqrt{q}) : \mathbb{Q}(\sqrt{p})] = 2$
$$\therefore [\mathbb{Q}(\sqrt{p}, \sqrt{q}) : \mathbb{Q}] = [\mathbb{Q}(\sqrt{p})(\sqrt{q}) : \mathbb{Q}(\sqrt{p})] \cdot [\mathbb{Q}(\sqrt{p}) : \mathbb{Q}]$$
$$\therefore [\mathbb{Q}(\sqrt{p}, \sqrt{q}) : \mathbb{Q}] = 4$$
}

\prob{4}{
  Let $n \geq 2$ and set $\epsilon := e^{\frac{2\pi i}{n}}$.
  \begin{enumerate}[(i)]
    \item Show that $\mathbb{Q}(\epsilon)$ is the splitting field for $x^n - 1$ over $\mathbb{Q}$.
    \item If $n = p$ is prime, find $[\mathbb{Q}(\epsilon) : \mathbb{Q}]$.
          Hint: first make an educated guess for the minimal polynomial of $\epsilon$
          over $\mathbb{Q}$, then show that the function $\phi : \mathbb{Q}[x] \rightarrow \mathbb{Q}[x]$ given by $\phi(f(x)) = f(x + 1)$ is an automorphism,
          and then apply/consider Eisenstein's criterion.
  \end{enumerate}
}{\\
  \admit
}



\end{document}
