% This template retrieved from "https://guides.nyu.edu/LaTeX/templates"

\documentclass[11pt]{article} 

\usepackage{geometry} 
\usepackage{amsmath}  
\usepackage{graphicx} 
\usepackage{amssymb}
\usepackage{amscd}
\usepackage{amsfonts}
\usepackage[shortlabels]{enumitem}

\newcommand{\prob}[3]{\begin{flushleft}
        \textbf{Problem #1}: \\
        #2 
		\textbf{Solution:} 
		#3

\end{flushleft}}

\newcommand{\makeHWtitle}[1]{
    \begin{center}
    \Large{Homework #1 - MATH 791} 
        \vspace{5pt}
        
        \normalsize{Will Thomas}
        \vspace{5pt}
    \end{center}
}

\begin{document}

\makeHWtitle{14}

\prob{1}{
    Let $F$ be a field. Follow (and prove each of) the steps given in the Lecture of February 24 to
    prove the Fundamental Theorem of Arithmetic to show that every monic polynomial with coefficients in
    $F$ can be factored uniquely as a product of monic, irreducible, polynomials with coefficients in $F$. \\
}{\\
    \textbf{Lemma}:

    Given $f, g_1, g_2$ all monic irreducible polynomials.
    $$f = g_1 \cdot g_2 \implies f \mid g_1 \lor f \mid g_2$$
    Proof:

    Since $f$ is irreducible $\implies f = g_1 \cdot g_2 \implies g_1$ is a unit or $g_2$ is a unit.

    Since $g_1$ and $g_2$ are monic, the only units they could be are $1$ themselves

    WLOG this implies that $f = g_1 * 1 \implies f = g_1$ and thus $f \mid g_1$ and $f\mid g_1 \lor f \mid g_2$ \\
    \textbf{End Proof}


    First, we will prove that any $x \in F[X]$ can be factored as a product of monic, irreducibles (not uniquely)

    Let us define
    $$X := \{ \text{monic polynomials that cannot be factored by monic,irreducibles} \}$$

    Since polynomials are well-ordered when you look at the degree, we can pick a least element $n \in X$.

    Since $n$ was the least element, if $n$ is irreducible, then it can be factored as itself trivially. If $n$ is not irreducible, then there exists a factorization such that $n = a * b$ where $a,b$ are not monic, irreducibles. However, since $a,b \notin X$, they themselves can be factored into monic, irreducibles $\implies a = \alpha_1 \cdots \alpha_n, \ b = \beta_1 \cdots \beta_m$. This can be combined to create a monic, irreducible factorization of $n = \alpha_1 \cdots \alpha_n \cdot \beta_1 \cdots \beta_m$
    $$\therefore \text{$X = \emptyset$ and any monic polynomial can be factored}$$

    Now to prove the uniqueness:

    Given $x \in F[X]$, where $x = f_1(x) \cdots f_r(x)$ and $x = g_1(x) \cdots g_s(x)$ where $f_i, g_i$ are all monic, irreducible polynomials.

    We want to show that after re-indexing, $f_i = g_i$ and $r = s$ (uniqueness up to ordering).

    Let us assume WLOG $r < s$ and set $e_i := deg(f_i)$ and $h_i := deg(g_i)$

    Induct on $n = e_1 + \cdots + e_r$:

    If $n = 1$ then $f_1(x) = g_1(x) \cdots g_s(x)$, we know by our Lemma that since $f_1$ is monic, irreducible that $f_1 \mid g_1 \lor f_1 \mid g_2 \cdots g_s$. Let us assume that the first case holds and $f_1 \mid g_1$ or we re-index to force this case to hold.

    Since $g_1$ is also monic, irreducible $\implies f_1 = g_1$ and $e_1 = h_1$. Additionally, this will force the rest of the $g_i = 1$ and they can then not be counted as part of the factorization, so $r = s = 1$.

    If $n > 1$ then $\implies f_1 \mid f_1 \cdots f_r \implies f_1 \mid g_1 \cdots g_s \implies f_1 \mid g_i$ for some $i$ (by a similar argument as the case when $n = 1$)

    Since $g_i$ is prime $\implies f_1 = g_i \implies$ after re-indexing $f_1 = g_1$ and
    $$f_1 \cdot f_2 \cdots f_r = f_1 \cdot g_2 \cdots g_s$$
    We can divide both sides by $f_1$ to get $f_2 \cdots f_r = g_2 \cdots g_s$, and then apply our induction hypothesis to solve the rest of the problem.

    $$\therefore \text{We can factor every monic polynomial in $F[X]$ uniquely as a product of irreducibles}$$
}

\prob{2}{
    Prove that repeated applications of the division algorithm can be used to find the GCD to $a,b \in \mathbb{Z}$, and
    that backwards substitution with the system of equations generated by this process gives $m, n \in \mathbb{Z}$ such that
    $\gcd(a,b) = ma + nb$ \\
}{ \\
    We will make the assumption that $a, b$ are positive, if not, re-index them to have a lower bounded finite subset of $\mathbb{Z}$ for well-foundedness. \\
    First, let us prove that the division algorithm will halt.

    Given $a, b \in \mathbb{Z}$ we have two cases WLOG $a < b$ or $a = b$. If $a = b$ this stops trivially

    If $a < b$, then we will get $\exists q, r \in \mathbb{Z}, \ b = q_0 * a + r$ for $0 \leq r < a$.
    This can be repeated by taking $a = q_1 * r + r_1$ for $0 \leq r_1 < r$. Since $\mathbb{Z}$ is well-founded, we know that this relation will eventually reach $0$ and halt.

    Next, we will prove that this method preserves the GCD.

    Let us assume that $r > 0$, and we know by induction that for $a = q_1 * r + r_1$ that the $\gcd(a,r) = x_1 = m_1 * a + n_1 * r$.
    We want to show that since $x_1 \mid a \land x_1 \mid r \implies x_1 \mid b$ and is the maximal such element that allows $x_1 \mid a \land x_1 \mid b$.

    Since $x_1 \mid a \implies a = x_1 * a'$ and $x_1 \mid r \implies r = x_1 * r'$. We can back-substitute this to get $b = q_0 * (x_1 * a') + x_1 * r'$ which can be factored to $b = x_1 * (q_0 * a' + r')$.
    $$\therefore x_1 \mid b$$
    To show that it is maximal, we need to show that no greater gcd could exist. Let us assume a greater gcd $x_1 < x'$ existed. Then $b = x' * b'$ and we also know that $a = x' * a'$. This can be combined into $x' * b' = q_0 * x' * a' + r \implies x' *(b' - q_0 * a') = r$.
    This leads to a contradiction as this implies that $x' \mid r$, but we assumed $x' > x_1$ and $x_1$ was the GCD for $r$
    $$\therefore \text{$x_1$ is the maximal element, thus it is the GCD}$$

    To show that we can use backwards substitution to reach the GCD value, we can use the same induction setup as before.

    We know that $\gcd(b,a) = \gcd(a,r) = m_1 * a + n_1 * r$ we also know that $b = q_0 * a + r \implies r = b - q_0 * a$
    $$\gcd(a,b) = m_1 * a + n_1 * (b - q_0 * a)$$
    $$ = m_1 * a + n_1 * b - n_1 * q_0 * a = (m_1 - n_1 * q_0) * a + n_1 * b$$
    Thus, using backwards substitution we can solve for the Bezout coefficients

}

\prob{3}{
    Use the Euclidean algorithm to find $\gcd(120, 54)$ and write the GCD as an integer combination of 120 and 54 as in Bezout's Principle. \\
}{ \\
    \begin{align*}
        120 & = 54 * 2 + 12 \\
        54  & = 12 * 4 + 6  \\
        12  & = 6 * 2 + 0
    \end{align*}
    So the $\gcd(120,54) = 6$, we can back-substitute to get
    $$6 = 1*54 - 4 * 12 = 1 * 54 - 4 * (1* 120 - 2 * 54) = 9 * 54 - 4 * 120$$
    $$6 = (9)(54) + (-4)(120)$$
    The Bezout coefficients are $9$ and $-4$
}


\end{document}
