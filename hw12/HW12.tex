% This template retrieved from "https://guides.nyu.edu/LaTeX/templates"

\documentclass[11pt]{article} 

\usepackage{geometry} 
\usepackage{amsmath}  
\usepackage{graphicx} 
\usepackage{amssymb}
\usepackage{amscd}
\usepackage{amsfonts}
\usepackage[shortlabels]{enumitem}

\newcommand{\prob}[3]{\begin{flushleft}
        \textbf{Problem #1}: \\
        #2 
		\textbf{Solution:} 
		#3

\end{flushleft}}

\newcommand{\makeHWtitle}[1]{
    \begin{center}
    \Large{Homework #1 - MATH 791} 
        \vspace{5pt}
        
        \normalsize{Will Thomas}
        \vspace{5pt}
    \end{center}
}

\begin{document}

\makeHWtitle{12}

\prob{1}{
    Let $S$ be any ring and $R$ denote the ring of $2 \times 2$ matrices over $S$. Prove that $I \subseteq R$ is a two-sided ideal if and only if there exists a two-sided ideal $J \subseteq S$ such that $I = M_2(J)$. \\
}{ \\
    $I \subseteq R$ is a two-sided ideal $\implies \exists \text{ two-sided ideal } J \subseteq S,\ s.t.\ I = M_2(J)$ \\
    Let us define
    $$J := \{ x \mid \begin{pmatrix}
            x & * \\
            * & *
        \end{pmatrix} \in I \}$$
    Using this, the final conclusion that $I = M_2(J)$, so we only need to prove that $J$ defined this way is a two-sided ideal.
    $$A = \begin{pmatrix}
            a_1 & a_2 \\
            a_3 & a_4
        \end{pmatrix}, \ B = \begin{pmatrix}
            b_1 & b_2 \\
            b_3 & b_4
        \end{pmatrix}, \ C = \begin{pmatrix}
            c_1 & c_2 \\
            c_3 & c_4
        \end{pmatrix}, A, C \in R,\ B \in I$$
    $$ABC \in I = \begin{pmatrix}
            a_1b_1c_1 + a_2b_3c_1 + a_1b_2c_3 + a_2b_4c_3 & * \\
            *                                             & *
        \end{pmatrix}$$
    $$\implies a_1b_1c_1 + a_2b_3c_1 + a_1b_2c_3 + a_2b_4c_3 \in J$$
    Which means that $J$ must be a two-sided ideal as for any arbitrary elements in $R$, the left and right sided conjugation will be in $I$ which makes it in $J$. \\

    Now for the converse: $\exists J \subseteq S, I = M_2(J) \implies I \subseteq R$ is a two-sided ideal \\
    Since $J$ is a two-sided ideal, that means that any elements $s_i, t_i \in S$, $s_ij_it_i \in J$ \\
    Leveraging this, we know that if we are given any other elements $A, C \in R$ where $R = M_2(S)$, that for $B \in I \subseteq R$
    $$ABC = \begin{pmatrix}
            a_1b_1c_1 + a_2b_3c_1 + a_1b_2c_3 + a_2b_4c_3 & a_1b_1c_2 + a_2b_3c_2 + a_1b_2c_4 + a_2b_4c_4 \\
            a_3b_1c_1 + a_4b_3c_1 + a_3b_2c_3 + a_4b_4c_3 & a_3b_1c_2 + a_4b_3c_2 + a_3b_2c_4 + a_4b_4c_4\end{pmatrix}$$
    $$= \begin{pmatrix}
            a_1b_1c_1 & a_1b_1c_2 \\
            a_3b_1c_1 & a_3b_1c_2
        \end{pmatrix} + \begin{pmatrix}
            a_2b_3c_1 & a_2b_3c_2 \\
            a_4b_3c_1 & a_4b_3c_2
        \end{pmatrix} + \begin{pmatrix}
            a_1b_2c_3 & a_1b_2c_4 \\
            a_3b_2c_2 & a_3b_2c_4
        \end{pmatrix} + \begin{pmatrix}
            a_2b_4c_3 & a_2b_4c_4 \\
            a_4b_4c_3 & a_4b_4c_4
        \end{pmatrix}$$
    Since each element of these mini-matrices is obvious in $J$, they are all elements of a two-sided ideal, and thus $I = M_2(J)$ is a two-sided ideal
}

\prob{2}{
    Let $R$ be a ring and $X \subseteq R$ be a subset. Define $\langle X \rangle$, the \emph{two-sided ideal generated by $X$} to be the intersection of all two-sided ideals of $R$ containing $X$. First, show that $\langle X \rangle$ is a two-sided ideal of $R$ containing $X$ and then show $\langle X \rangle$ is the set of all finite expressions of the form $r_1x_1s_1 + \cdots + r_nx_ns_n$, with $r_i,s_j \in R$ and $x_i \in X$. \\
}{ \\
    We have two cases, either $X$ is a two-sided ideal itself, or it needs to be extended to $X' := X \cup \{ x'_1, \ldots, x'_m \}$ to be a two-sided ideal. \\
    If $X$ is a two-sided ideal itself, then the intersection of all two-sided ideals of $R$ containing $X$ will be some $\langle X \rangle = \bigcap (X \cup \cdots)$ and $X$ will be the minimal element containing $X$, thus $\langle X \rangle$ will be the two-sided ideal generated by $X$ and the intersection. \\
    If $X$ has to be extended to $X'$, then that means $\exists x_k \in X$ such that $rx_ks \notin X$ for all $r, s \in R$, and that the additional elements $x'_1, \ldots, x'_m$ all $x_k \in X'$. \\
    Any other two-sided ideal of $R$ containing $X$ will also have to contain $x'_1, \ldots, x'_m$ as well since they are what allow $x_k \in X'$
    $$\implies \text{Any other two-sided ideal $H_i$ will have to contain $X'$}$$
    $$\therefore \langle X \rangle := \bigcap (X' \cup H_i \cdots)$$
    So any two-sided ideal generated by $X$ will be the intersection of all two-sided ideals of $R$ containing $X$
}

\prob{3}{
    Let $R$ and $S$ be rings. Let $R \times S$ denote $\{(r,s) \mid r \in R \text{ and } s \in S \}$.
    \begin{enumerate}[(i)]
        \item Show that $R \times S$ is a ring under coordinate-wise addition and multiplication.
        \item Show that $K \subseteq R \times S$ is a two-sided ideal if and only if $K = I \times J$, for $I$ a two-sided ideal in $R$ and $J$ a two-sided ideal in $S$.
    \end{enumerate}
}{
    \begin{enumerate}[(i)]
        \item This seems very obvious, define $+*$ as $(r_1, s_1) +* (r_2, s_2) = (r_1 + r_2, s_1 + s_2)$ which will always be in $R \times S$. Very similarly for $\times*$. \\
              All other features should be directly inheirited except possibly the Abelian group feature. This becomes obvious though since it is $+*$ which is coordinate-wise, so the ring's $R$ and $S$ apply their own Abelian features.
              $$\therefore R\times S \text{ is a ring}$$

        \item Let us presume that $I$ is not a two-sided ideal, this means that $\exists r,r' \in R$ such that $ri_1r' \notin I$, then the point $(i_1, j) \in K$ will not allow $(r, s) \times* (i_1, j) \times* (r', s') = (ri_1r', sjs') \notin K$.
              A very similar argument can be applied to $J$
              $$\therefore K \text{ is a two-sided ideal $\iff K = I \times J$ where $I$ and $J$ are two-sided ideals}$$
    \end{enumerate}
}

\prob{4}{
    Let $R$ and $S$ be a commutative rings, so that $T := R \times S$ is also a commutative ring. Set $e_1 = (1,0)$ and $e_2 = (0,1)$
    \begin{enumerate}[(i)]
        \item Show that $Te_1 := \{ te_1 \mid t \in T \}$ is both an ideal of $T$ and a ring, in its own right. Similarly, for $Te_2$.
        \item $T = Te_1 + Te_2$ and $Te_1 \cap Te_2 = 0$
        \item Show that $T$ is isomorphic to $Te_1 \times Te_2$
    \end{enumerate}
}{
    \begin{enumerate}[(i)]
        \item $Te_1$ will be an ideal of $T$ if for any elements $(a1,b1), (a2,b2) \in T$, $(a1,b1)(t1,0)(a2,b2) \in Te_1$. This reduces to $(a1t1a2,0) \in Te_1$ since $Te_1$ is define over all $t \in T$. Similar proof for $Te_2$ \\
              To see that it is a ring, we just recognize that we can create an isomorphism between $Te_1 \cong R$ and $R$ is a commutative ring, thus $Te_1$ will be a commutative ring. Similar for $Te_2$ except for the comm. ring $S$. (Pick a trivial homomorphism $\phi((r,0)) = r$ or $\phi((0,s)) = s$)

        \item Any element $t \in T, t = (t1,t2) = t1 \in Te_1 + t2 \in Te_2$. \\
              Let us presume an element $(t1,t2) \neq (0,0) \in Te_1 \cap Te_2 \implies (t1,t2) \in Te_1 \implies t2 = 0 \land (t1,t2) \in Te_2 \implies t1 = 0$ thus no element other than $0 \in Te_1 \cap Te_2$

        \item As we showed in part (i), $Te_1 \cong R$ and $Te_2 \cong S$ which can be combined to show $T \cong Te_1 \times Te_2$ just by direct application of the definition of $T$
    \end{enumerate}
}

\prob{5}{
    Let $R$ be a commutative ring. An element $e \in R$ is called an \emph{idempotent} if $e^2 = e$. We say that $e$ is a \emph{non-trivial idempotent} if $e \neq 0,1$.
    \begin{enumerate}[(i)]
        \item Supposed that $e \in R$ is a non-trivial idempotent. Show that $1 - e$ is also a non-trivial idempotent and $e \cdot (1 - e) = 0$.
        \item Show that $Re$ is both an ideal and a ring. Similarly for $R(1-e)$.
        \item Show that $Re \cap R(1-e) = 0$
        \item Show that $R$ is isomorphic to $Re \times R(1-e)$
    \end{enumerate}
}{
    \begin{enumerate}[(i)]
        \item If $e$ is non-trivial idempotent that means $e^2 = e$, if we take $(1-e)^2 = 1 - 2e + e^2 = (1 - e)$ so it is also a non-trivial idempotent. \\
              $e\cdot(1-e) = e - e^2 = e - e = 0$

        \item $Re$ will be an ideal if for any $r, s \in R$, $e' \in Re$, $re's \in Re$ since it is a commutative ring $rse' \in Re$ and since $e' = r'e$ we get $rsr'e \in Re$ and since $rsr' \in R$ we know this must hold. This ring property will hold as well since we have shown it is an ideal so the abelian subgroup holds, and multiplication will work by the argument over a comm. ring. \\
              Reasoning about $R(1-e)$, $r, s \in R$, $e' \in R(1-e)$, $re's \in R(1-e)$ since it is comm. ring $rse' \in R(1-e)$ and $e' = r'(1-e)$ we get $rsr'(1-e) \in R(1-e)$ and $rsr' \in R$ trivially. Same argument for why it is a ring.

        \item If $x \in Re \cap R(1-e)$ then that means that $\forall r \in R, rx \in Re$ and $rx \in R(1-e)$, however if we pick $r = 1$ then $x \in Re$ and $x \in R(1-e)$. \\
              This means that $\exists r_1 \in R, x = r_1e$ and $\exists r_2 \in R, x = r_2(1-e)$
              Setting these two equal $r_1e = r_2 - r_2e \implies (r_1 + r_2)e = r_2 \implies
                  r_1r_2^{-1}e + e = 0 \implies (r_1r_2^{-1} - 1)e = 0 \implies r_1r_2^{-1} = e^{-1} + 1 = (1 - e) = xr_2^{-1} \implies x = r_1 = r_2 \implies x = 0$

        \item We know that any element $r \in R$ will be in either, so we can form this isomorphism by a cardinality argument (this obviously needs a lot of work).
    \end{enumerate}
}



\end{document}
