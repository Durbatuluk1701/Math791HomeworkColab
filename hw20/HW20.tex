% This template retrieved from "https://guides.nyu.edu/LaTeX/templates"

\documentclass[11pt]{article} 

\usepackage{geometry} 
\usepackage{amsmath}  
\usepackage{graphicx} 
\usepackage{amssymb}
\usepackage{amscd}
\usepackage{amsfonts}
\usepackage[shortlabels]{enumitem}

\newcommand{\prob}[3]{\begin{flushleft}
        \textbf{Problem #1}: \\
        #2 
		\textbf{Solution:} 
		#3

\end{flushleft}}

\newcommand{\admit}{
  \begin{flushright}
    \textbf{Admitted}
  \end{flushright}
}

\newcommand{\B}[1]{%bold
	\mathbb{#1}
}

\newcommand{\makeHWtitle}[1]{
    \begin{center}
    \Large{Homework #1 - MATH 791} 
        \vspace{5pt}
        
        \normalsize{Will Thomas}
        \vspace{5pt}
    \end{center}
}

\begin{document}

\makeHWtitle{20}
Throughout this assignment $R$ denotes a commutative ring.

\prob{1}{
An ideal $P \neq R$ is said to be a $\emph{prime ideal}$ if for $a, b \in R$, $ab \in P \Rightarrow a \in P $ or $b \in P$. Prove that $P$ is a prime ideal iff $R/P$ is an integral domain.\\
}
{\\
For the first direction, $R/P$ is an ID $\Rightarrow$ $P$ is a prime ideal:
\begin{align*}
&\text{Suppose } r_1r_2 + P = 0_{R/P} = P \Rightarrow r_1r_2 \in P\\
&r_1r_2 + P = (r_1 + P)(r_2 + P) = 0_{R/P} = P \\
&\Rightarrow (r_1 + P) = 0_{R/P} = P or (r_2 + P) = 0_{R/P} = P \text{ Since $R/P$ is an ID}\\ 
&\\
&\text{If $r_1 + P = P$}:\\
&\text{then } r_1 \in P\\
&\text{And if $r_2 + P = P$}:\\
&\text{then } r_2 \in P\\
\end{align*}
So $r_1r_2 \in P \Rightarrow r_1 \in P$ or $r_2 \in P$, so $P$ is a prime ideal. Now the other direction:
\begin{align*}
&\text{Suppose $P$ is a prime ideal}\\
&\text{Suppose $r_1r_2 \in P$}\\
&\Rightarrow r_1r_2 + P = P = 0_{R/P} = (r_1 + P)(r_2 + P)\\
&\text{Either $r_1 \in P$ or $r_2 \in P$ since $P$ is a prime ideal}\\
&\Rightarrow (r_1 + P) = 0_{R/P} \text{ or } (r_2 + P) = 0_{R/P}\\
\end{align*}
So if $P$ is a prime ideal, then $r_1r_2 + P = 0_{R/P} \Rightarrow (r_1 + P) = 0_{R/P} $ or $(r_2 + P) = 0_{R/P}$, so $R/P$ is an integral domain.
}

\prob{2}{
An ideal $M \neq R$ is a $\emph{maximal ideal}$ if whenever $J \subseteq R$ is an ideal satisfying $M \subseteq J \subseteq R$, then $J = M$ or $J = R$. In other words, $M$ is maximal among the proper ideals of $R$. It follows from Zorn's Lemma, that if $I \subset R$ is an ideal, then there exists a maximal ideal $M \subseteq R$ with $I \subseteq M$. In particular, every commutative ring has at least one maximal ideal. Prove that $M$ is a maximal ideal iff $R/M$ is a field. Conclude that every maximal ideal is a prime ideal, and give an example of a prime ideal that is not a maximal ideal.
\\
}
{\\
First we can prove that $M$ being maximal implies that $R/M$ is a field. We have to show that every element in $R/M$ other than the additive identity has a multiplicative inverse.

\begin{align*}
&\text{let } r + M \in R/M\\
&\text{If } r \in M \Rightarrow r + M = M\text { is the additive identity and doesn't have a multiplicative inverse}\\
&\text{If } r \notin M \Rightarrow (\langle r \rangle + M) \supset M\\
&\Rightarrow (\langle r \rangle + M) = R \text{ since $M$ is maximal}\\
&\Rightarrow 1 \in R, 1 \in (\langle r \rangle + M)\\
&\Rightarrow 1 = rr' + m'\\
&\Rightarrow 1 + (-m') = rr'\\
&\Rightarrow rr' \in M + 1 \Rightarrow rr' = m_2 + 1\\
&\Rightarrow rr' + M = 1 + M\\
&\Rightarrow (r + M)(r' + M) = (1 + M)\\
&\text{So $(r + M)$ has the multiplicative inverse $(r' + M)$, so $R/M$ is a field}\\
\end{align*}
Now for the other direction, we show that $R/M$ being a field implies that $M$ is maximal.

\begin{align*}
&\text{$R/M$ is a field, so every nonzero element has a multiplicative inverse}\\
&\text{Suppose } J \supset M \text{is an ideal that is a strict superset of $M$}\\
&\exists j \in J, j \notin M\\
&\text{(j + M) has an inverse :}\\
&(j + M)(r' + M) = (1 + M)\\
&\Rightarrow jr' + M = 1 + M\\
&\Rightarrow jr' + m' = 1 \text{ for some $m'$}\\
&m' = jr'' \text{ since $M \subset J$}\\
& = jr' + jr'' = 1 = j(r' + r'') = 1\\
&\Rightarrow j(r'+r'')*r = r \text{ for all $r \in R$}\\
\end{align*}
So $J = R$, which means that $M$ is a maximal ideal.
\newline
We can notice that all fields are integral domains. Suppose F is a field.
\begin{align*}
&a, b \in F\\
&ab = 0\\
&\text{If $a = 0$ we are done. If $a \neq 0$, it has an inverse}\\
&ab = 0\\
&a^{-1}ab = a^{-1}0\\
&b = 0\\
\end{align*}
So either $a$ or $b$ are the additive inverse. So if $M$ is a maximal ideal:
\begin{align*}
& M \text{ maximal} \Rightarrow R/M \text{ is a field} \Rightarrow R/M \text{ is an ID} \Rightarrow M \text{ is a prime ideal from problem 1}\\
&M \text{ maximal} \Rightarrow M \text{ is a prime ideal}\\
\end{align*}

Finally, consider the ideal $\langle 0 \rangle$ in the ring $\B{Z}$. $\B{Z}$ is an ID, so if $ab \in <0> \Rightarrow ab = 0 \Rightarrow a \in <0> \text{or } b \in <0>$. But $<0> \subseteq <r>$ for all $r \in R$, so it is not maximal.
}

\prob{3}{
Let $R$ be a commutative ring. Ideals $I, J \subseteq R$ are said to be comaximal if $I + J = R$. Prove that $I$ and $J$ are comaximal iff there is no maximal ideal $M$ containing both $I$ and $J$.
\\
}
{\\
First we can show that if $I + J = R$, then there is no maximal ideal that is a superset of both $I$ and $J$. 
\begin{align*}
&I+J = R\\
&\Rightarrow i + j = 1 \text{ For some $i \in I, j \in J$}\\
&\\
&\text{Suppose $M$ is an ideal, and that } I \subseteq M, J \subseteq M\\
&i + j \in M \Rightarrow 1 \in M \Rightarrow r \in M\\
&\Rightarrow M = R\\
\end{align*}
Since $M$ must be the whole ring, there is no maximal ideal containing both $I$ and $J$. Now the other direction, where we assume there is no maximal ideal containing $I$ and $J$:
\begin{align*}
&\text{If $I \subset M, J \subset M$, and $M$ is an ideal, then } M = R\\
&\text{Let } M = I + J\\
&I + J \text{ Is an ideal, which we can show through closure under addition, multiplication, etc.}\\
&\text{It inherits associativity and distributivity from $R$}\\
&I \subseteq I + J, J \subseteq I + J \Rightarrow I+J = R
\end{align*}
So $I, J$ must be comaximal.
}

\prob{4}{
Suppose $I, J$ are comaximal ideals in the commutative ring $R$. Show that $I \cap J = IJ$.
\\
}
{\\
First we can show that if $a \in IJ \Rightarrow a \in I \cap J$:
\begin{align*}
&a \in IJ\\
&a = r_1i_1j_1r_1' + ... + r_ki_kj_kr_k'\\
& = i_1(r_1j_1r_1') + ... + i_k(r_kj_kr_k') \in I\\
& = j_1(r_1i_1r_1') + ... + j_k(r_ki_kr_k') \in J\\
&\Rightarrow r_1i_1j_1r_1' + ... + r_ki_kj_kr_k' \in I \cap J\\
\end{align*}
Now the other direction:
\begin{align*}
&a \in I \cap J\\
&\text{We use the fact that } I + J = R \Rightarrow 1 = i' + j'\\
&a = a*1 = a(i' + j') = ai' + aj' = i'a + aj'\\
& = i'j_a + i_aj' \text{ since $a$ can be considered an element of both $I$ and $J$}\\
& = i'j_a + i_aj' \in IJ
\end{align*}
So $I \cap J = IJ$\\
}


\prob{5}{
For $I$ and $J$ as in $4$, prove that the natural map $\phi: R \Rightarrow (R/I) \times (R/J)$ given by $\phi (r) = (r + I, r + J)$ is a surjective ring homomorphism whose kernel equals $I \cap J$. Conclude that $R/IJ \cong (R/I) \times (R/J)$. When $R = \B{Z}$, this isomorphism is one version of the $\emph{Chinese Remainder Theorem}$.
\\
}
{\\
First we should show that $\phi$ is a surjective homomorphism:
\begin{align*}
&\phi(a_1) + \phi(a_2) = (a_1 + I, a_1 + J) + (a_2 + I, a_2 + J)\\
& = (a_1 + a_2 + I, a_1 + a_2 + J)\\
& = \phi(a_1+a_2)\\
&\\
&\phi(a_1)*\phi(a_2) = (a_1 + I, a_1 + J)*(a_2 + I, a_2 + J)\\
& = (a_1a_2 + I, a_1a_2 + J) = \phi(a_1a_2)\\
\end{align*}
Now we have to show that the homomorphism is surjective:
\begin{align*}
&\text{Since $I, J $ are comaximal:}\\
&I + J = R \Rightarrow 1 = i' + j' \text{ for some $i' \in I, j' \in J$}\\
&\text{So for any $r \in R$}:\\
&r = ri' + rj'\\
&\Rightarrow \forall r \in R\\
&r + I = ri' + rj' + I = rj' + I \text{ because $ri' \in I$}\\
&\text{Similarly,}\\
&r + J = ri' + rj' + J = ri' + J \text{ because $ri' \in I$}\\
&\\
&\text{For any element in } (R / I) \times (R /J ), (a_1 + I, a_2 + J)\\
&\text{Let } r \in R, r = a_2i' + a_1j', \text{ with $i', j'$ as above}\\
&\\
&\phi(r) = (a_2i' + a_1j' + I, a_2i' + a_1j' + J)\\
&a_2i' \in I, a_1j' \in J\\
&\Rightarrow \phi(r) = (a_1j' + I, a_2i' + J) = (a_1 + I, a_2 + J) \text{ from what we proved earlier}
\end{align*}

\begin{align*}
&\text{Suppose } k \in R \text{ s.t. } \phi(k) = (I, J) \\
&\text{ ((I, J) is the additive identity of the ring $(R/I) \times (R/J)$)}\\
&\phi(k) = (k + I, k + J) = (I, J)\\
&\Rightarrow k \in I, k \in J \Rightarrow k \in I \cap J\\
&\text{So } ker \phi = I \cap J\\
\end{align*}

From the first isomorphism theorem, we see that
\begin{align*}
&R/(I\cap J) \cong (R/I) \times (R/J)\\
&\text{But } I \cap J = IJ \text{ since $I, J$ are comaximal}\\
&\Rightarrow R/(IJ) \cong (R/I) \times (R/J)
\end{align*}

}

\end{document}
