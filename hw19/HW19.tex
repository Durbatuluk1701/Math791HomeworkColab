% This template retrieved from "https://guides.nyu.edu/LaTeX/templates"

\documentclass[11pt]{article} 

\usepackage{geometry} 
\usepackage{amsmath}  
\usepackage{graphicx} 
\usepackage{amssymb}
\usepackage{amscd}
\usepackage{amsfonts}
\usepackage[shortlabels]{enumitem}

\newcommand{\prob}[3]{\begin{flushleft}
        \textbf{Problem #1}: \\
        #2 
		\textbf{Solution:} 
		#3

\end{flushleft}}

\newcommand{\admit}{
  \begin{flushright}
    \textbf{Admitted}
  \end{flushright}
}

\newcommand{\makeHWtitle}[1]{
    \begin{center}
    \Large{Homework #1 - MATH 791} 
        \vspace{5pt}
        
        \normalsize{Will Thomas}
        \vspace{5pt}
    \end{center}
}

\begin{document}

\makeHWtitle{19}
Throught this assignment $R$ denotes a commutative ring.

\prob{1}{
  Let $I \subseteq R$ be an indeal, and $R[x]$ denote the polynomial ring in $x$ over $R$. Let $I[x]$ denote the set of polynomials in $R$ with coefficients in $I$ and let $\langle I \rangle$ denote the ideal of $R[x]$ generated by the set $I$.
  Show that $I[x] = \langle I \rangle$. \\
}{ \\
  Admitted
}

\prob{2}{
Maintaining the notation from 1, show that the rings $R[x]/I[x]$ and $(R/I)[x]$ are isomorphic. \\
}{\\
Admitted
}

\prob{3}{
  Let $R[[x]]$ denot the formal power series ring over $R$, i.e., the set of expressions of the form $\sum_{i = 0}^{\infty}a_ix^i$, with $a_i \in R$.
  Note this is purely an algebraic expression and does not involve any notion of convergence.
  We add and multiply element of $R[[x]]$ in the expected way:
  If $f = \sum_{i=0}^{\infty}a_ix^i$ and $g = \sum_{i=0}^{\infty}b_ix^i$, then:
  $f + g = \sum_{i=0}^{\infty}(a_i + b_i)x^i$ and $fg = \sum_{k=0}^{\infty}c_kx^k$, where $c_k = \sum_{i + j = k}a_ib_j$. For $I \subseteq R$, let $I[[x]]$ denote the elements in $R[[x]]$, all of whose coefficients belong to $I$.
  \begin{enumerate}[(i)]
    \item Verify that $R[[x]]$ is a ring and $I[[x]]$ is an ideal of $R[[x]]$.
    \item Show that if $I$ is finitely generated, then $\langle I \rangle = I[[x]]$ as ideals of $R[[x]]$.
    \item Can you give an example where $I[[x]] \neq \langle I \rangle$
  \end{enumerate}
}{ \\
  Admitted
}
Here is Eisentsteain's Criterion, which is an important test for irreducibility of polynomials over a UFD. \\
\textbf{Eisenstein's Criterion:} Let $R$ be a UFD with quotient field $K$.
Suppose $f(x) = a_nx^n + \cdots + a_0 \in R[x]$ is a primitive polynomial.
Let $p \in R$ be a prime element and suppose: (i) $p \mid a_i$, for all $0 \leq i < n$.
(ii) $p \nmid a_n$, and (iii) $p^2 \nmid a_0$.
Then $f(x)$ is irreducible over $K$ (equivalently, over $R$).
For example, $x^6 + 10x^2 + 5x + 15$ is irreducible over $\mathbb{Q}$, by using Eisenstein's criterion and $p = 5$.

\prob{4}{
  Let $p \in \mathbb{Z}$ be prime and $f_p(x) = x^{p-1} + x^{p-2} + \cdots + x + x \in \mathbb{Z}[x]$. Use Eisenstein's criterion, together with the following fact to show that $f_p(x)$ is irreducible over $\mathbb{Q}[x]$: $f_p(x)$ is irreducible over $\mathbb{Q}$ if and only if $f_p(x + 1)$ is irreducible over $\mathbb{Q}$. \\
}{ \\
  Admitted
}

\prob{5}{
  Use Eisenstein's criterion and the fact that $\mathbb{Q}[x]$ is a UFD to show that $x^2 + y^2 - 9$ is irreducible in $\mathbb{Q}[x,y]$. \\
}{ \\
  Admitted.
}



\end{document}
