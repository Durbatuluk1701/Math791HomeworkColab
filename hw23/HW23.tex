% This template retrieved from "https://guides.nyu.edu/LaTeX/templates"

\documentclass[11pt]{article} 

\usepackage{geometry} 
\usepackage{amsmath}  
\usepackage{graphicx} 
\usepackage{amssymb}
\usepackage{amscd}
\usepackage{amsfonts}
\usepackage[shortlabels]{enumitem}

\newcommand{\prob}[3]{\begin{flushleft}
        \textbf{Problem #1}: \\
        #2 
		\textbf{Solution:} 
		#3

\end{flushleft}}

\newcommand{\admit}{
  \begin{flushright}
    \textbf{Admitted}
  \end{flushright}
}

\newcommand{\lr}[1]{
  \langle #1 \rangle
}

\newcommand{\makeHWtitle}[1]{
    \begin{center}
    \Large{Homework #1 - MATH 791} 
        \vspace{5pt}
        
        \normalsize{Will Thomas}
        \vspace{5pt}
    \end{center}
}

\begin{document}

\makeHWtitle{23}

\prob{1}{
  Show that $p(x) = x^3 + x^2 + 2x + 1$ is irreducible over $\mathbb{Z}_3$. \\
}{ \\
  Let us assume for contradiction that we could factor out some $(x - \alpha)$ for $\alpha \in \mathbb{Z}$.

  Then we know that $x^3 + x^2 + 2x + 1 = (x - \alpha)(\beta_2 x^2 + \beta_1 x + \beta_0)$.
  Let us just destruct on possible values of $\alpha$

  \textbf{$\alpha = 0$}: \\
  Then clearly this will not work as $(x)(\beta_2 x^2 + \beta_1 x + \beta_0)$ will have no constant term, but one is required.

  \textbf{$\alpha = 1$}: \\
  Then we will have $(x - 1)(\beta_2 x^2 + \beta_1 x + \beta_0)$
  and the constant term will force $\beta_0 = -1 \equiv 2$.

  We will also know that $\beta_2 = 1$ (from monic polynomial).
  This forces
  $$(x - 1)(x^2 + \beta_1 x + 2) = x^3 + (2 + \beta_1 x^2) + (-\beta_1 x) + 1$$
  $$= x^3 + x^2 + 2x + 1$$
  This equality is irreconcilable for any possible $\beta_1$

  \textbf{$\alpha = 2$}: \\
  Then we will have $(x - 2)(\beta_2 x^2 + \beta_1 x + \beta_0)$
  and the constant term will force $\beta_0 = -2 \equiv 1$.


  We will also know that $\beta_2 = 1$ (from monic polynomial).
  This forces
  $$(x - 2)(x^2 + \beta_1 x + 1) = x^3 + (1 + \beta_1 x^2) + (-2*\beta_1 x) + 1$$
  $$= x^3 + x^2 + 2x + 1$$
  This equality is irreconcilable for any possible $\beta_1$

  $$\therefore \text{$p(x)$ is irreducible over $\mathbb{Z}_3$}$$
}

\prob{2}{
  For $p(x)$ as in the previous problem, from class we know that there is a field $K$ containing $\mathbb{Z}_3$ and $\alpha \in K$ such that $p(\alpha) = 0$.
  \begin{enumerate}[(i)]
    \item How many elements are in the field $\mathbb{Z}_3(\alpha)$?
    \item In the field $\mathbb{Z}_3(\alpha)$ calculate $A \cdot B$ and $A^{-1}$ for
          $A := 1 + 2 \alpha + \alpha^2$ and $B := 2 + \alpha + 2 \alpha^2$
  \end{enumerate}
}{
  \begin{enumerate}[(i)]
    \item We know that $p(x)$ is irreducible, and that its degree is $3$
          $$\therefore [\mathbb{Z}_3(\alpha) : \mathbb{Z}_3] = 3$$
          We know that $\{ 0,1,2 \} \in \mathbb{Z}_3$, and the basis for
          $\mathbb{Z}_3(\alpha)$ contains $1$.

          That means we have two extra elements in our basis and the closure of it would be the cross so $3^3 = 27$

          Thus, there are 27 elements in $\mathbb{Z}_3(\alpha)$

    \item $$A \cdot B = 2 + 4\alpha + 2\alpha^2 + \alpha + 2\alpha^2 + \alpha^3 + 2 \alpha^2 + 4\alpha^3 + 2 \alpha^4$$
          $$A \cdot B = 2 + 2\alpha + 2\alpha^3 + 2\alpha^4$$
          $$A \cdot B = 2(1 + \alpha)(1 + \alpha^3)$$

          As for $A^{-1}$, I do not want to calculate this.
  \end{enumerate}
}

\prob{3}{
  Give an example of a field with 125 elements. \\
}{ \\
  Take two primes $p,q$ and add them to $\mathbb{Z}_5$
  to get $\mathbb{Z}_5(p,q)$.
}

\prob{4}{
Fix a prime $p$. Assume that for all $n \geq 1$, there exists an irreducible polynomial in $\mathbb{Z}_p[x]$ having degree $n$. Show that for all primes $p$ and $n \geq 1$, there exists a field with $p^n$ elements. \\
}{ \\
Using similar constructions as the past 2 problems,
we can always just take the irreducible polynomial $p(x) \in \mathbb{Z}_p[x]$ with degree $n$, such that $p(p) = 0$.
Then $|\mathbb{Z}_p[x](p)| = p^n$
}

\prob{5}{
  Let $\alpha \in K \supseteq \mathbb{Z}_2$ be a root of $x^2 + x + 1$.
  Show that $\mathbb{Z}_2(\alpha)$ is the splitting field for $x^2 + x + 1$. \\
}{\\
  We know that $x^2 + x + 1$ is irreducible, and if $\alpha$ is a root of $p(x) = x^2 + x + 1$, it is immediately known that $\mathbb{Z}_2(\alpha)$ is the splitting field of $x^2 + x + 1$.
}

\end{document}
