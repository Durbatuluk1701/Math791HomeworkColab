% This template retrieved from "https://guides.nyu.edu/LaTeX/templates"

\documentclass[11pt]{article} 

\usepackage{geometry} 
\usepackage{amsmath}  
\usepackage{graphicx} 
\usepackage{amssymb}
\usepackage{amscd}
\usepackage{amsfonts}
\usepackage[shortlabels]{enumitem}

\newcommand{\prob}[3]{\begin{flushleft}
        \textbf{Problem #1}: \\
        #2 
		\textbf{Solution:} 
		#3

\end{flushleft}}

\newcommand{\admit}{
  \begin{flushright}
    \textbf{Admitted}
  \end{flushright}
}

\newcommand{\B}[1]{%bold
	\mathbb{#1}
}

\newcommand{\st}{
	\text{ s.t. }
}

\newcommand{\makeHWtitle}[1]{
    \begin{center}
    \Large{Homework #1 - MATH 791} 
        \vspace{5pt}
        
        \normalsize{Will Thomas}
        \vspace{5pt}
    \end{center}
}

\begin{document}

\makeHWtitle{18}

The problems in this homework set deal with a special kind of PID.
Let $R$ be a principal ideal domain with the property that, given any two prime elements, $\pi_1$ and $\pi_2$, $\langle \pi_1 \rangle = \langle \pi_2 \rangle$, i.e., up to a unit multiple, there is just one prime element, say $\pi \in R$.
Such a ring is called a \emph{discrete valuation ring}, denoted DVR, and $\pi \in R$ is called a \emph{uniformizing parameter}.

\prob{1}{
  Fix a prime $p \in \mathbb{Z}$. Let $R$ denote the set of rational numbers whose denominators is not divisible by $p$.
  First show that $R$ is a subring of $\mathbb{Q}$, and then show that $R$ is a DVR with uniformizing parameter $p$. \\
}{ \\
First we need to show that $R$ is a subring of $\mathbb{Q}$. $R$ inherits associativity and distributivity from $\mathbb{Q}$, so we only need to show that $(R, +)$ is a group and that $R$ is closed under multiplication.
\newline
Closure of $(R, +)$ under composition:
\begin{align*}
&\frac{a_1}{a_2}, \frac{b_1}{b_2} \in R\\
&p \nmid a_2, p \nmid b_2\\
&\frac{a_1}{a_2} + \frac{b_1}{b_2} = \frac{a_1b_2+ b_1a_2}{a_2b_2}\\
\end{align*}
Because $\B{Z}$ is a UFD, we use the contrapositive of one of the requirements of a prime to say that $p \nmid a_2, p \nmid b_2 \Rightarrow p \nmid a_2b_2$.
\begin{align*}
&\Rightarrow \frac{a_1b_2+ b_1a_2}{a_2b_2} \in R\\
\end{align*}
Closure of $(R, +)$ under inverses:
\begin{align*}
&\frac{a_1}{a_2} \in R\\
&\left( - \frac{a_1}{a_2} \right) = \frac{-a_1}{a_2} = \frac{a_1}{-a_2} \in R\\
\end{align*}

Closure of $R$ under multiplication:
\begin{align*}
&\frac{a_1}{a_2}, \frac{b_1}{b_2} \in R\\
&\frac{a_1}{a_2}* \frac{b_1}{b_2} = \frac{a_1b_1}{a_2b_2}\\
&p \nmid a_2, p \nmid b_2 \Rightarrow p \nmid a_2b_2\\
&\Rightarrow \frac{a_1}{a_2}* \frac{b_1}{b_2} \in R
\end{align*}
Also $R$ contains the multiplicative identity $\frac{1}{1}$.
\newline
Now we need to show that $R$ is a DVR with uniformizing parameter $p$.
\newline
Since primes can't be units, they must be elements of $R$ without a multiplicative inverse. An element has no multiplicative inverse iff it is a multiple of $p$.
\begin{align*}
&\frac{a_1p}{a_2} \in R, p \nmid a_2\\
&\text{We assume WLOG that $p \nmid a_1$}\\
&\\
&\text{If $\frac{a_1p}{a_2}$ had an inverse $\frac{x_1}{x_2}$:}\\
&\frac{a_1p}{a_2}*\frac{x_1}{x_2} = \frac{a_1px_1}{a_2x_2} \in \left[ \left( \frac{1}{1} \right) \right]\\
& p \nmid x_2, p \nmid a_2 \Rightarrow p \nmid a_2x_2\\
&\text{But } p \mid pa_1x_1 \\
&\text{So $p$ divides the numerator but not the denominator}\\
&\Rightarrow \frac{a_1p}{a_2}*\frac{x_1}{x_2} \notin \left[ \left( \frac{1}{1} \right) \right]\\
&\text{So a multiple of $p$ does not have an inverse}\\
&\\
&\text{If an element in $R$ is not a multiple of $p$, then it has an inverse}\\
&\frac{a_1}{a_2} \in R, p \nmid a_1, \nmid a_2\\
&\Rightarrow \frac{a_1}{a_2}\frac{a_2}{a_1} = 1, \frac{a_2}{a_1} \in R
\end{align*}
So for any non-unit prime $\frac{pa_1}{a_2}$:
\begin{align*}
&p \mid \frac{pa_1}{a_2}\\
&\frac{pa_1}{a_2} * \frac{a_2}{a_1} = p \Rightarrow \frac{pa_1}{a_2} \mid p\\
&\Rightarrow \langle \frac{pa_1}{a_2} \rangle = \langle p \rangle
\end{align*}
}
\prob{2}{
  Let $R$ be a DVR with uniformizing paramter $\pi \in R$. Show that $\bigcap_{n \geq 1} \langle \pi^n \rangle = 0$. \\
}{ \\
  First we can see that since $0 \in \langle \pi ^n \rangle$ for all $n$, $0 \in \bigcap_{n \geq 1} \langle \pi ^ n \rangle$.
  \newline
  Now consider a nonzero element $a \in R$. We know that $a$ can be written as a finite product of irreducibles (proved earlier), and that $\pi$ is prime, therefore irreducible, so
  
\begin{align*}
&\text{case 1: } \pi \nmid a, \text{ or}\\
&\text{case 2: } a = \pi ^s b\\
&\text{Suppose there are two ways of writing $a$ in case 2:}\\
&a = \pi ^ m b = \pi ^ n b' \text{ suppose $m \geq n$, and $m, n \geq 1$}\\
&\pi ^ {m-n}b = b' \text{ From cancellation in IDs}\\
&\Rightarrow \pi ^ n b' = \pi^n(\pi ^{m-n} b)\\
&\text{So the two factorizations have the same number of $\pi$'s} \\
&\\
&\text{In case 1:}\\
&\pi \nmid a \\
&\Rightarrow \forall q, \pi q \neq a \Rightarrow a \notin \langle \pi \rangle\\
&\\
&\text{In case 2:}\\
&a = \pi ^ s b\\
&\text{WLOG, assume $\pi \nmid b$}\\
&\Rightarrow \forall q, \pi q \neq b\\
&\Rightarrow \forall q, \pi^{s+1} q \neq a\\
&\Rightarrow a \notin \langle \pi ^ {s+1} \rangle
\end{align*}
In all cases, for any nonzero $a$ there exists an ideal of a power of $\pi$ such that $a \notin \langle \pi ^ n \rangle$. So since $a$ is not in all ideals of the form $\langle \pi ^ n \rangle $, $a \notin \bigcap _{n \geq 1} \langle \pi ^n \rangle$. But $0 \in \bigcap _{n \geq 1}$. So $\bigcap _{n \geq 1} = 0$
}

\prob{3}{
  Let $R$ be a DVR with uniformizing parameter $\pi \in R$. Show that every element in $R$ can be written uniquely as $u\pi^n$ for some $n \geq 0$ and $u \in R$ a unit.
  Conclude that if $K$ dnotes the quotient field of $R$, then every element in $K$ can be written uniquely in the form $u \pi^n$ for some $n \in \mathbb{Z}$ and $u \in R$, a unit. \\
}{ \\
  First we can prove that $R$ is a UFD. We know that every element in $R$ can be written as a product of irreducibles. Now we prove that irreducible elements generate maximal ideals:
\begin{align*}
&p \in R \text{ is irreducible}\\
&\text{Suppose } <p> \subseteq <j> \subseteq R\\
&\Rightarrow p = jr\\
&\text{$p$ is irreducible } \Rightarrow \text{$j $ or $r$ is a unit}\\
&\text{if $r$ is a unit, then } <j> = <p>\\
&\text{if $j$ is a unit, then } <j> = R \text{ which is not an ideal}\\
\end{align*}
So $<p>$ is a maximal ideal.
Now we can prove that maximal ideals are prime ideals, and that an element generating a prime ideal is prime.
\begin{align*}
&\text{Suppose $<q>$ is a maximal ideal and that }\\
&q = ab\\
&\text{We also assume that $q \mid ab, q \nmid a$}\\
&<q> \subset <q> + <a>\\
&<q> + <a> \text{ is an ideal, since the sum of ideals is an ideal}\\
&<q> + <a> = R, \text{because it is a strict superset of $<q>$, and $<q>$ is maximal}\\
&\\
&1 \in <q> + <a>\\
&\Rightarrow 1 = r_1q + r_2a\\
&\Rightarrow b*1 = b*r_1q + b*r_2a\\
&\Rightarrow b = br_1q + r_2(ab)\\
&q \mid (br_1)q, q \mid (r_2)ab\\
&\Rightarrow q \mid br_1q + r_2(ab) \Rightarrow q \mid b
\end{align*}
So $q$ is prime. We have shown that every irreducible element is prime in $R$. So every element can be written as a finite product of irreducibles, and therefore can be written as a finite product of primes. This implies that $R$ is a UFD, which includes uniqueness of the factorizations. But since there is only one prime:
\begin{align*}
&r \in R \Rightarrow r = (u_0\pi)(u_1\pi)(u_2\pi)...(u_k\pi)\\
&\Rightarrow r = (u_0...u_k)\pi^k = u'\pi^k\\
\end{align*}

Now we have to prove that every element of $K$ can be written in a similar way.
\begin{align*}
&\frac{a}{b} \in K\\
&\frac{a}{b} = \frac{u_1\pi ^m}{u_2 \pi ^n}\\
&\text{If $m \geq n$:}\\
&\frac{u_1\pi ^m}{u_2 \pi ^n} = \frac{u_2 \pi ^ n u_2^{-1}u_1\pi ^{m-n}}{u_2 \pi ^n}\\
& = \frac{u_2^{-1}u_1\pi ^{m-n}}{1} = \frac{u'\pi ^{m-n}}{1} = u'\pi ^{m-n}\\
&\\
&\text{If $m < n$:}\\
&\frac{u_1\pi ^m}{u_2 \pi ^n} = \frac{u_1\pi ^m}{u_1 \pi ^m u_1^{-1} u_2 \pi ^{n-m}}\\
&=\frac{1}{u_1^{-1} u_2 \pi ^{n-m}} = \frac{1}{u' \pi ^{n-m}}\\
& = \left( \frac{u' \pi ^{n-m}}{1} \right)^{-1}\\
& = \left( u' \pi ^{n-m} \right)^{-1} \text{ using the convention of writing $\frac{a}{1} = a$ in K}\\
&= u' \left( \pi ^{n-m} \right) ^{-1}
\end{align*}
}

\prob{4}{
  Let $R$ be a DVR with uniformizing parameter $\pi \in R$, and quotient field $K$. Define $v : K \rightarrow \mathbb{Z} \cup \{ \infty \}$ by $v(0) = \infty$ and for $\alpha \neq 0, v(\alpha) =n$, where $\alpha \in K$ and $\alpha = u \pi^n$, as in 3. Show that for all $\alpha, \beta \in K$:
  \begin{enumerate}[(i)]
    \item $v(\alpha + \beta) \geq min\{ v(\alpha), v(\beta) \}$
    \item $v(\alpha \beta) = v(\alpha) + v(\beta)$
  \end{enumerate}
  Observe that $R = \{ a \in K \mid v(a) \geq 0 \}$ \\
}{ \\
  Proof of $(i)$:
  \newline
\begin{align*}
&\text{let } \alpha = u_1 \pi ^ {n_1}, \beta = u_2 \pi ^{n_2}\\
&\text{We are considering the elements as being in $K$, but $u_1, u_2$ are units in $R$ and $\alpha, \beta \in R$}\\
&\text{Assume that $n_1 < n_2$}\\
&u_1 \pi ^ {n_1} + u_2 \pi ^{n_2} = u_3 \pi ^ {n_3} \text{ from 3.}\\
&\\
&\text{Suppose that $n_3 < n_1$ (This will show a contradiction)}\\
&u_1 \pi ^ {n_1} + u_2 \pi ^{n_2} = u_3 \pi ^{ n_3}\\
& = \pi ^ {n_1}(u_1 + u_2 \pi ^{n_2-n_1}) = \pi ^{n_1} u_3 \pi ^ {n_3-n_1}\\
& = u_1 + u_2 \pi ^{n_2-n_1} = u_3 \pi ^ {n_3-n_1}\\
&\text{Note that $n_3-n_1 < 0$}\\
& = u_1 + u_2 \pi ^{n_2-n_1} = u_3 \left( \pi ^ {n_1-n_3} \right)^{-1}\\
&\text{The expression on the left side is an element of $R$ from closure}\\
&\text{This implies the right side is in $R$ (its not)}\\
&\text{But we need to check that $u_3 \left( \pi ^ {n_1-n_3} \right)^{-1}$ can't be in $R$}\\
&\text{If $u_3 \left( \pi ^ {n_1-n_3} \right)^{-1} \in R$, then it can be written $u_1 \pi ^b, b \geq 0$}\\
&\Rightarrow u_1 \pi ^b * u_3 \left( \pi ^ {n_1-n_3} \right) = 1\\
&\Rightarrow A * u_1 \pi ^b * u_3 \left( \pi ^ {n_1-n_3} \right) = A\\
&\text{This contradicts the unique factorization of $A$ in $R$}\\
&\text{Because of the contradiction we can conclude that $n_3 \geq \min (n_1, n_2)$}
\end{align*}
Proof of $(ii)$:
\begin{align*}
&\text{let } \alpha = u_1 \pi ^ {n_1}, \beta = u_2 \pi ^{n_2}\\
&\alpha \beta = u_1 \pi ^ {n_1} u_2 \pi ^{n_2}\\
&= u_1 u_2 \pi ^ {n_1} \pi ^{n_2} = u_1 u_2 \pi ^ {n_1+n_2}\\
& = u' \pi ^ {n_1+n_2}\\
&\Rightarrow v(\alpha \beta) = n_1 + n_2 = v(\alpha ) + v(\beta)\\
\end{align*}
}

\prob{5}{
  Let $K$ be a field. Suppose $v : K \rightarrow \mathbb{Z} \cup \{ \infty \}$ is a function such that for all $\alpha, \beta \in K$:
  \begin{enumerate}[(i)]
    \item $v(\alpha) = \infty \iff \alpha = 0$
    \item $v(\alpha + \beta) \geq min\{ v(\alpha), v(\beta) \}$
    \item $v(\alpha \beta) = v(\alpha) + v(\beta)$
  \end{enumerate}

  Such a function is called a \emph{discrete valuation} on $K$.
  We assume that $v$ takes values other than $0$ and $\infty$.
  Set $R := \{ \alpha \in K \mid v(\alpha) \geq 0 \}$. Prove that $R$ is DVR by the following steps below:
  \begin{enumerate}[(i)]
    \item Show that $u \in R$ is a unit $\iff v(u) = 0$. Hint: First show $v(1) = 0$.
    \item Show there exist element $r \in R$, with $v(r) > 0$.
    \item Prove that if $r \in R$, and $v(r) > 0$, then as an element of $K$, $v(\frac{1}{r}) = -v(r)$.
    \item Suppose $c := \min\{v(r) \mid r \in R \text{ and } v(r) > 0 \}$. Show that the image of $v$ is $c \mathbb{Z}$.
    \item Show that if $\pi \in R$ and $v(\pi) = c$, then $R$ is a DVR with uniformizing parameter $\pi$.
  \end{enumerate}
}
{\\
Proof of $(i)$:
\begin{align*}
&u \in R \text{ is a unit}\\
&\Rightarrow u u ^{-1} = 1\\
&\Rightarrow v(u) + v(u^{-1}) = v(1) = 0\\
& u, u^{-1} \in R \Rightarrow v(u), v(u^{-1}) \geq 0\\
&\Rightarrow v(u) = 0, v(u^{-1}) = 0\\
&\\
&\text{Now assume $v(b) = 0$ for some $b \in R$}\\
&b^{-1} \in K = \frac{1}{b}\\
&\Rightarrow b * \frac{1}{b} = 1 \text{ in K}\\
&\Rightarrow v(b) + v(\frac{1}{b}) = v(1) = 0\\
&v(b) = 0 \Rightarrow v(\frac{1}{b}) = 0\\
&\text{since $v(\frac{1}{b}) \geq 0$, $\frac{1}{b} = b^{-1} \in R$}\\
&\text{So $b$ has a multiplicative inverse in $R \Rightarrow $ $b$ is a unit}\\
\end{align*}

Proof of $(ii)$:
\newline
We assumed that the function $v(\alpha)$ takes values other than $0$ and $\infty$, so for some $\alpha \in K, v(\alpha) \neq 0$. We need to show that there exists an element in $R$ with the same property.
\begin{align*}
&\alpha \in K, v(\alpha) = c, c \neq 0\\
&\text{if $c > 0$, then $c \in R$ by definition of $R$, and we are done)}\\
&\text{if $c < 0$, then $c^{-1} \in K$}\\
&c * c^{-1} = 1\\
&v(c*c^{-1}) = v(1) = 0\\
&v(c) + v(c^{-1}) = 0\\
&v(c^{-1}) = -v(c)\\
&\Rightarrow v(c^{-1} > 0)\\
&\text{So $c^{-1} \in R$ and we are done}
\end{align*}

Proof of $(iii)$:
\newline
\begin{align*}
&r \in R, v(r) > 0\\
&\text{The multiplicative inverse of $r$ in $K$ is $\frac{1}{r}$}\\
&\frac{1}{r}*r = 1 \text{ In K}\\
&\Rightarrow v(\frac{1}{r}*r) = v(1)\\
&\Rightarrow v(\frac{1}{r})+v(r) = 0\\
&\Rightarrow v(\frac{1}{r}) = -v(r)\\
\end{align*}

Proof of $(iv)$:
\newline
First we can show that all of the elements in the image of $v$ are divisible by $c$. Then we can show that for each multiple of $c$, $ac$, with $a, c \in \B{Z}$, there exists an element $t \st v(t) = ac$ .

\begin{align*}
&\exists r_0 \in R \st v(r_0) = c\\
&\text{Suppose there is an element $r \in R \st c \nmid v(r)$}\\
&v(r) = qc + r', 0 < r' < c \text{ or}\\
&v(r) = q(v(r_0)) + r', 0 < r' < c\\
&\text{Consider the elements of $K$, $r, \frac{1}{r_0}$}\\
&\text{let } b = r * \frac{1}{r_0} * \frac{1}{r_0} * ... * \frac{1}{r_0}\\
&\text{Where there are $q$ terms of $\frac{1}{r_0}$}\\
&v(b) = v(r * \frac{1}{r_0} * \frac{1}{r_0} * ... * \frac{1}{r_0})\\
&\Rightarrow v(b) = v(r) - v(r_0) - ... - v(r_0)\\
&\Rightarrow v(b) = v(r) - qv(r_0) = r'\\
&r' > 0, \text{ so } b \in R. \text{ but $r' < c$, and $c := \min \{v(r) \mid r \in R$\}}\\
&\text{This is a contradiction, so every number in the image of $v$ must be divisible by c}\\
&\\
\end{align*}
Now we have to show that each multiple of $c$ is in the range of $v$, with $K$ as the domain.
\begin{align*}
&\text{For each value $qc$ when $q \geq 0$}:\\
&r_0*r_0*...*r_0 \in K \text{ with $q$ terms of $r_0$}\\
&v(r_0*r_0*...*r_0) = q*v(r_0) = qc, \text{ So there exists an element $r$ in $K$ \st $v(r) = qc$}\\
&\\
&\text{If $q < 0$}:\\
&\frac{1}{r_0}*\frac{1}{r_0}*...*\frac{1}{r_0} \in K \text{ with $q$ terms of $\frac{1}{r_0}$}\\
&v(\frac{1}{r_0}*\frac{1}{r_0}*...*\frac{1}{r_0}) = \vert q \vert *v(\frac{1}{r_0}) = \vert q \vert * (-v(r_0)) = q*v(r_0) = qc\\
&\text{ So there exists an element $r = \frac{1}{r_0}*\frac{1}{r_0}*...*\frac{1}{r_0}$ in $K$ \st $v(r) = qc$}\\
&\\
\end{align*}
So if an element is in $c \B{Z} $, then that element is in $im(v)$, and if an element is not in $c\B{Z}$, then it is not in $im(v)$. This proves the sets are equal.


Proof of $(v)$:
To prove that $R$ is a DVR, we show that for each prime $p \in R$, $v(p) = v(\pi) \Rightarrow p = u\pi$. So $\langle p \rangle = \langle \pi \rangle$.
\begin{align*}
&\text{Let $p$ be a prime in $R$}\\
&\Rightarrow p \text{ is irreducible, so } p = ab \Rightarrow \text{$a$ or $b$ is a unit}\\
&\Rightarrow v(p) = v(a) + v(b) \text{ So one term on the right is nonzero}\\
&\\
&\text{We know from part $(iv)$ that for all $r \in R$, $c \mid v(r)$}\\
&\text{And the image of $v$ with $R$ as the domain is $c\B{Z}^+$}\\
&c \mid v(p)\\
&\Rightarrow v(\pi)*q = v(p)\\
&\Rightarrow v(\pi) + ... + v(\pi) = v(p)\\
&\text{But $p$ is irreducible, so only one of the $q$ summands on the left is nonzero (by induction)}\\
&\Rightarrow v(\pi) = v(p)\\
&\Rightarrow \pi = up\\
&\Rightarrow \pi \mid p, p \mid \pi \Rightarrow \langle p \rangle \subseteq \langle \pi \rangle, \langle \pi \rangle \subseteq \langle p \rangle,\\
&\Rightarrow \langle \pi \rangle = \langle p \rangle
\end{align*}
So $R$ is a DVR with uniformizing parameter $p$.

}
\end{document}
