% This template retrieved from "https://guides.nyu.edu/LaTeX/templates"

\documentclass[11pt]{article} 

\usepackage{geometry} 
\usepackage{amsmath}  
\usepackage{graphicx} 
\usepackage{amssymb}
\usepackage{amscd}
\usepackage{amsfonts}
\usepackage[shortlabels]{enumitem}

\newcommand{\prob}[3]{\begin{flushleft}
        \textbf{Problem #1}: \\
        #2 
		\textbf{Solution:} 
		#3

\end{flushleft}}

\newcommand{\admit}{
  \begin{flushright}
    \textbf{Admitted}
  \end{flushright}
}

\newcommand{\B}[1]{%bold
	\mathbb{#1}
}

\newcommand{\makeHWtitle}[1]{
    \begin{center}
    \Large{Homework #1 - MATH 791} 
        \vspace{5pt}
        
        \normalsize{Will Thomas}
        \vspace{5pt}
    \end{center}
}

\begin{document}

\makeHWtitle{18}

The problems in this homework set deal with a special kind of PID.
Let $R$ be a principal ideal domain with the property that, given any two prime elements, $\pi_1$ and $\pi_2$, $\langle \pi_1 \rangle = \langle \pi_2 \rangle$, i.e., up to a unit multiple, there is just one prime element, say $\pi \in R$.
Such a ring is called a \emph{discrete valuation ring}, denoted DVR, and $\pi \in R$ is called a \emph{uniformizing parameter}.

\prob{1}{
  Fix a prime $p \in \mathbb{Z}$. Let $R$ denote the set of rational numbers whose denominators is not divisible by $p$.
  First show that $R$ is a subring of $\mathbb{Q}$, and then show that $R$ is a DVR with uniformizing parameter $p$. \\
}{ \\
First we need to show that $R$ is a subring of $\mathbb{Q}$. $R$ inherits associativity and distributivity from $\mathbb{Q}$, so we only need to show that $(R, +)$ is a group and that $R$ is closed under multiplication.
\newline
Closure of $(R, +)$ under composition:
\begin{align*}
&\frac{a_1}{a_2}, \frac{b_1}{b_2} \in R\\
&p \nmid a_2, p \nmid b_2\\
&\frac{a_1}{a_2} + \frac{b_1}{b_2} = \frac{a_1b_2+ b_1a_2}{a_2b_2}\\
\end{align*}
Because $\B{Z}$ is a UFD, we use the contrapositive of one of the requirements of a prime to say that $p \nmid a_2, p \nmid b_2 \Rightarrow p \nmid a_2b_2$.
\begin{align*}
&\Rightarrow \frac{a_1b_2+ b_1a_2}{a_2b_2} \in R\\
\end{align*}
Closure of $(R, +)$ under inverses:
\begin{align*}
&\frac{a_1}{a_2} \in R\\
&\left( - \frac{a_1}{a_2} \right) = \frac{-a_1}{a_2} = \frac{a_1}{-a_2} \in R\\
\end{align*}

Closure of $R$ under multiplication:
\begin{align*}
&\frac{a_1}{a_2}, \frac{b_1}{b_2} \in R\\
&\frac{a_1}{a_2}* \frac{b_1}{b_2} = \frac{a_1b_1}{a_2b_2}\\
&p \nmid a_2, p \nmid b_2 \Rightarrow p \nmid a_2b_2\\
&\Rightarrow \frac{a_1}{a_2}* \frac{b_1}{b_2} \in R
\end{align*}
Also $R$ contains the multiplicative identity $\frac{1}{1}$.
\newline
Now we need to show that $R$ is a DVR with uniformizing parameter $p$.
\newline
Since primes can't be units, they must be elements of $R$ without a multiplicative inverse. An element has no multiplicative inverse iff it is a multiple of $p$.
\begin{align*}
&\frac{a_1p}{a_2} \in R, p \nmid a_2\\
&\text{We assume WLOG that $p \nmid a_1$}\\
&\\
&\text{If $\frac{a_1p}{a_2}$ had an inverse $\frac{x_1}{x_2}$:}\\
&\frac{a_1p}{a_2}*\frac{x_1}{x_2} = \frac{a_1px_1}{a_2x_2} \in \left[ \left( \frac{1}{1} \right) \right]\\
& p \nmid x_2, p \nmid a_2 \Rightarrow p \nmid a_2x_2\\
&\text{But } p \mid pa_1x_1 \\
&\text{So $p$ divides the numerator but not the denominator}\\
&\Rightarrow \frac{a_1p}{a_2}*\frac{x_1}{x_2} \notin \left[ \left( \frac{1}{1} \right) \right]\\
&\text{So a multiple of $p$ does not have an inverse}\\
&\\
&\text{If an element in $R$ is not a multiple of $p$, then it has an inverse}\\
&\frac{a_1}{a_2} \in R, p \nmid a_1, \nmid a_2\\
&\Rightarrow \frac{a_1}{a_2}\frac{a_2}{a_1} = 1, \frac{a_2}{a_1} \in R
\end{align*}
So for any non-unit prime $\frac{pa_1}{a_2}$:
\begin{align*}
&p \mid \frac{pa_1}{a_2}\\
&\frac{pa_1}{a_2} * \frac{a_2}{a_1} = p \Rightarrow \frac{pa_1}{a_2} \mid p\\
&\Rightarrow \langle \frac{pa_1}{a_2} \rangle = \langle p \rangle
\end{align*}
}
\prob{2}{
  Let $R$ be a DVR with uniformizing paramter $\pi \in R$. Show that $\bigcap_{n \geq 1} \langle \pi^n \rangle = 0$. \\
}{ \\
  \admit
}

\prob{3}{
  Let $R$ be a DVR with uniformizing parameter $\pi \in R$. Show that every element in $R$ can be written uniquely as $u\pi^n$ for some $n \geq 0$ and $u \in R$ a unit.
  Conclude that if $K$ dnotes the quotient field of $R$, then every element in $K$ can be written uniquely in the form $u \pi^n$ for some $n \in \mathbb{Z}$ and $u \in R$, a unit. \\
}{ \\
  \admit
}

\prob{4}{
  Let $R$ be a DVR with uniformizing parameter $\pi \in R$, and quotient field $K$. Define $v : K \rightarrow \mathbb{Z} \cup \{ \infty \}$ by $v(0) = \infty$ and for $\alpha \neq 0, v(\alpha) =n$, where $\alpha \in K$ and $\alpha = u \pi^n$, as in 3. Show that for all $\alpha, \beta \in K$:
  \begin{enumerate}[(i)]
    \item $v(\alpha + \beta) \geq min\{ v(\alpha), v(\beta) \}$
    \item $v(\alpha \beta) = v(\alpha) + v(\beta)$
  \end{enumerate}
  Observe that $R = \{ a \in K \mid v(a) \geq 0 \}$ \\
}{ \\
  \admit
}

\prob{5}{
  Let $K$ be a field. Suppose $v : K \rightarrow \mathbb{Z} \cup \{ \infty \}$ is a function such that for all $\alpha, \beta \in K$:
  \begin{enumerate}[(i)]
    \item $v(\alpha) = \infty \iff \alpha = 0$
    \item $v(\alpha + \beta) \geq min\{ v(\alpha), v(\beta) \}$
    \item $v(\alpha \beta) = v(\alpha) + v(\beta)$
  \end{enumerate}

  Such a function is called a \emph{discrete valuation} on $K$.
  We assume that $v$ takes values other than $0$ and $\infty$.
  Set $R := \{ \alpha \in K \mid v(\alpha) \geq 0 \}$. Prove that $R$ is DVR by the following steps below:
  \begin{enumerate}[(i)]
    \item Show that $u \in R$ is a unit $\iff v(u) = 0$. Hint: First show $v(1) = 0$.
    \item Show there exist element $r \in R$, with $v(r) > 0$.
    \item Prove that if $r \in R$, and $v(r) > 0$, then as an element of $K$, $v(\frac{1}{r}) = -v(r)$.
    \item Suppose $c := \min\{v(r) \mid r \in R \text{ and } v(r) > 0 \}$. Show that the image of $v$ is $c \mathbb{Z}$.
    \item Show that if $\pi \in R$ and $v(\pi) = c$, then $R$ is a DVR with uniformizing parameter $\pi$.
  \end{enumerate}
}

\end{document}
