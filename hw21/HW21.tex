% This template retrieved from "https://guides.nyu.edu/LaTeX/templates"

\documentclass[11pt]{article} 

\usepackage{geometry} 
\usepackage{amsmath}  
\usepackage{graphicx} 
\usepackage{amssymb}
\usepackage{amscd}
\usepackage{amsfonts}
\usepackage[shortlabels]{enumitem}

\newcommand{\prob}[3]{\begin{flushleft}
        \textbf{Problem #1}: \\
        #2 
		\textbf{Solution:} 
		#3

\end{flushleft}}

\newcommand{\admit}{
  \begin{flushright}
    \textbf{Admitted}
  \end{flushright}
}

\newcommand{\lr}[1]{
  \langle #1 \rangle
}

\newcommand{\makeHWtitle}[1]{
    \begin{center}
    \Large{Homework #1 - MATH 791} 
        \vspace{5pt}
        
        \normalsize{Will Thomas}
        \vspace{5pt}
    \end{center}
}

\begin{document}

\makeHWtitle{21}

\prob{1}{
  Prove that $1, \sqrt[3]{2}, \sqrt[3]{4} \in \mathbb{Q}(\sqrt[3]{2})$
  are linearly independent over $\mathbb{Q}$.
  Thus, $[\mathbb{Q}(\sqrt[3]{2}) : \mathbb{Q}] = 3$. \\
}{\\
  Let us start with $1, \sqrt[3]{2}$ and assume for contradiction that they are not linearly independent.

  That would mean that $\exists \lambda_1, \lambda_2 \in \mathbb{Q}$ such that $\lambda_1 * 1 + \lambda_2 * \sqrt[3]{2} = 0$.
  We can reduce this to $\lambda_1 + \lambda_2 * \sqrt[3]{2} = 0$, and since both $\lambda_1, \lambda_2 \in \mathbb{Q}$, we can factor out $\lambda_2$ from both to get.
  $$\lambda_2 * (\lambda_1' + \sqrt[3]{2}) = 0$$
  $$\implies \sqrt[3]{2} = - \lambda_1' \implies 2 = - (\lambda_1')^3$$
  This would mean that there exists a $\lambda_1' \in \mathbb{Q}, \lambda_1' = \sqrt[3]{-2}$
  which is a contradiction.
  $$\therefore 1, \sqrt[3]{2} \text{ are linearly independent}$$
  A very similar argument can be made when adding $\sqrt[3]{4}$ to the mix
  $$\therefore 1, \sqrt[3]{2}, \sqrt[3]{4} \text{ are linearly independent over $\mathbb{Q}$}$$

  Now to conclude that $[\mathbb{Q}[\sqrt[3]{2}] : \mathbb{Q}] = 3$, we need only realize that
  any element in $\mathbb{Q}[\sqrt[3]{2}]$ can be made of $1, \sqrt[3]{2}, \sqrt[3]{4}$.
  Thus the degree of $\mathbb{Q}[\sqrt[3]{2}]$ over $\mathbb{Q}$ is $3$.
}

\prob{2}{
  Find the multiplicative inverse of $1 + 2 \sqrt[3]{2}$ in $\mathbb{Q}(\sqrt[3]{2})$. \\
}{\\
  The multiplicative inverse will be $(x + y \sqrt[3]{2} + z \sqrt[3]{4})$ such that
  $$(1 + 2 \sqrt[3]{2})(x + y \sqrt[3]{2} + z \sqrt[3]{4}) = 1$$
  $$\implies (x + 4z) + (y + 2x) \sqrt[3]{2} + (z + 2y) \sqrt[3]{4} = 1$$
  We could solve this via
  $$\begin{pmatrix}
      1 & 0 & 4 & 1 \\
      2 & 1 & 0 & 0 \\
      0 & 2 & 1 & 0
    \end{pmatrix} \leadsto
    \begin{pmatrix}
      1 & 0 & 0 & \frac{1}{17}  \\
      0 & 1 & 0 & -\frac{2}{17} \\
      0 & 0 & 1 & \frac{4}{17}
    \end{pmatrix}$$
  Thus the inverse element is $(\frac{1}{17} - \frac{2}{17} \sqrt[3]{2} + \frac{4}{17} \sqrt[3]{4})$
}

\prob{3}{
  Can you write down the multiplicative inverse of $1 + \sqrt[3]{2} + \sqrt[3]{4}$ in $\mathbb{Q}(\sqrt[3]{2})$ without doing any calculations? \\
}{\\
  No I cannot, how am I supposed to without calculations.

  Using calculations, we see $(x + 2y + 2z) + (x + y + 2z) \sqrt[3]{2} + (x + y + z) \sqrt[3]{4} = 1$.
  $$\begin{pmatrix}
      1 & 2 & 2 & 1 \\
      1 & 1 & 1 & 0 \\
      1 & 1 & 1 & 0
    \end{pmatrix} \leadsto
    \begin{pmatrix}
      1 & 0 & 0 & -1 \\
      0 & 1 & 0 & 1  \\
      0 & 0 & 1 & 0
    \end{pmatrix}$$
  Thus, the inverse is $(-1 + 1 \sqrt[3]{2} )$
}

\prob{4}{
  Let $F := \mathbb{Q}(\sqrt{2})$. Define $K := F(\sqrt{3})$ to be the set $\{ a + b \sqrt{3} \mid a, b \in F \}$. Show that $[K : F] = 2$.
  Can you guess $[K : \mathbb{Q}]$? If so, give a proof validating your guess. \\
}{\\
  I would guess that $[K : \mathbb{Q}] = 4$.
  This is because $\dim_{\mathbb{Q}} K = \dim_{\mathbb{Q}} F(\sqrt{3})$
  And $F := \{ a + b \sqrt{2} \mid a,b \in \mathbb{Q} \}$, so
  $F(\sqrt{3}) := \{ a + b \sqrt{3} \mid a,b \in F \} = \{ a + b \sqrt{2} + c \sqrt{3} + d \sqrt{6} \mid a,b,c,d \in \mathbb{Q} \}$
  To find a basis for this in $\mathbb{Q}$, we need the elements $1, \sqrt{2}, \sqrt{3}, \sqrt{6}$
  $$\therefore [K : \mathbb{Q}] = 4$$
}

\prob{5}{
Let $p(x) = x^2 + x + 1 \in \mathbb{Z}_2[x]$.
\begin{enumerate}[(i)]
  \item Show that $p(x)$ is irreducible over $\mathbb{Z}_2$.
  \item Show the the commutative ring $\mathbb{Z}_2[x] / \lr{p(x)}$ has just four elements.
  \item Prove that the ring $\mathbb{Z}_2[x] / \lr{p(x)}$ is a field.
\end{enumerate}
}{
\begin{enumerate}[(i)]
  \item Let us assume it can be reduced $p(x) = f(x)g(x)$ and neither $f,g$ are units.
        So $deg(f(x)), deg(g(x)) = 1$.
        Additionally, we know that both $f(x)$ and $g(x)$ must have a constant term $1$, since $p(x)$ has a constant term $1$.
        So $f(x) = \alpha x + 1$ and $g(x) = \beta x + 1$
        $f(x) g(x) = \alpha \beta x^2 + (\alpha + \beta) x + 1$
        We know that $\alpha \beta = 1$, and since we are in $\mathbb{Z}_2$, that means $\alpha = \beta = 1$. However, this would cause the $x$ term to be $2x \equiv 0x$ which is not allow.
        $$\therefore \text{$p(x)$ is irreducible}$$

  \item Let us try to enumerate the possible elements of $F := \mathbb{Z}_2[x] / \lr{p(x)}$.
        We know that for every power of $x$, the coefficient is either $0$ or $1$, since we are in $\mathbb{Z}_2$.
        We can find the following elements in $F$: $\{ 1, x, x + 1, x^2 + 1, x^2 + x \}$.
        Any other element will be canceled out be the quotient $\lr{p(x)}$.
        To demonstrate this, take an element $z = a_n x^n + \cdots + a_0$ for $n \geq 3$.
        If $a_n = 1$ then take out $z' = z - p(x) * x^{n - 2} = (a_{n - 1} - 1)x^{n - 1} + \cdots + a_0$. Repeat this process until $n = 2$. At which point either $z' = p(x)$, in which case it has been canceled out, or $z' \in$ out previously laid out elements.
        \admit

  \item To show that it is a field, we need to show primarily
        the multiplicative inverse property. The other properties are inherited from (ii) stating that $\mathbb{Z}_2 / \lr{p(x)}$ is a commutative ring.
        Given the elements in $F$, we can find a multiplicative inverse for each.
        \begin{enumerate}
          \item $1$ is its own inverse
          \item $x * (x^2 + 1) = x^3 + x \equiv$
        \end{enumerate}
        \admit
\end{enumerate}
}


\end{document}
