% This template retrieved from "https://guides.nyu.edu/LaTeX/templates"

\documentclass[11pt]{article} 

\usepackage{geometry} 
\usepackage{amsmath}  
\usepackage{graphicx} 
\usepackage{amssymb}
\usepackage{amscd}
\usepackage{amsfonts}
\usepackage[shortlabels]{enumitem}

\newcommand{\prob}[3]{\begin{flushleft}
        \textbf{Problem #1}: \\
        #2 
		\textbf{Solution:} 
		#3

\end{flushleft}}

\newcommand{\admit}{
  \begin{flushright}
    \textbf{Admitted}
  \end{flushright}
}

\newcommand{\lr}[1]{
  \langle #1 \rangle
}

\newcommand{\makeHWtitle}[1]{
    \begin{center}
    \Large{Homework #1 - MATH 791} 
        \vspace{5pt}
        
        \normalsize{Will Thomas}
        \vspace{5pt}
    \end{center}
}

\begin{document}

\makeHWtitle{21}

\prob{1}{
  Prove that $1, \sqrt[3]{2}, \sqrt[3]{4} \in \mathbb{Q}(\sqrt[3]{2})$
  are linearly independent over $\mathbb{Q}$.
  Thus, $[\mathbb{Q}(\sqrt[3]{2}) : \mathbb{Q}] = 3$. \\
}{\\
  Let us start with $1, \sqrt[3]{2}$ and assume for contradiction that they are not linearly independent.

  That would mean that $\exists \lambda_1, \lambda_2 \in \mathbb{Q}$ such that $\lambda_1 * 1 + \lambda_2 * \sqrt[3]{2} = 0$.
  We can reduce this to $\lambda_1 + \lambda_2 * \sqrt[3]{2} = 0$, and since both $\lambda_1, \lambda_2 \in \mathbb{Q}$, we can factor out $\lambda_2$ from both to get.
  $$\lambda_2 * (\lambda_1' + \sqrt[3]{2}) = 0$$
  $$\implies \sqrt[3]{2} = - \lambda_1' \implies 2 = - (\lambda_1')^3$$
  This would mean that there exists a $\lambda_1' \in \mathbb{Q}, \lambda_1' = \sqrt[3]{-2}$
  which is a contradiction.
  $$\therefore 1, \sqrt[3]{2} \text{ are linearly independent}$$
  A very similar argument can be made when adding $\sqrt[3]{4}$ to the mix
  $$\therefore 1, \sqrt[3]{2}, \sqrt[3]{4} \text{ are linearly independent over $\mathbb{Q}$}$$

  Now to conclude that $[\mathbb{Q}[\sqrt[3]{2}] : \mathbb{Q}] = 3$, we need only realize that
  any element in $\mathbb{Q}[\sqrt[3]{2}]$ can be made of $1, \sqrt[3]{2}, \sqrt[3]{4}$.
  Thus the degree of $\mathbb{Q}[\sqrt[3]{2}]$ over $\mathbb{Q}$ is $3$.
}

\prob{2}{
  Find the multiplicative inverse of $1 + 2 \sqrt[3]{2}$ in $\mathbb{Q}(\sqrt[3]{2})$. \\
}{\\
  Admitted
}

\prob{3}{
  Can you write down the multiplicative inverse of $1 + \sqrt[3]{2} + \sqrt[3]{4}$ in $\mathbb{Q}(\sqrt[3]{2})$ without doing any calculations? \\
}{\\
  Admitted
}

\prob{4}{
  Let $F := \mathbb{Q}(\sqrt{2})$. Define $K := F(\sqrt{3})$ to be the set $\{ a + b \sqrt{3} \mid a, b \in F \}$. Show that $[K : F] = 2$.
  Can you guess $[K : \mathbb{Q}]$? If so, give a proof validating your guess. \\
}{\\
  Admitted
}

\prob{5}{
Let $p(x) = x^2 + x + 1 \in \mathbb{Z}_2[x]$.
\begin{enumerate}[(i)]
  \item Show that $p(x)$ is irreducible over $\mathbb{Z}_2$.
  \item Show the the commutative ring $\mathbb{Z}_2[x] / \lr{p(x)}$ has just four elements.
  \item Prove that the ring $\mathbb{Z}_2[x] / \lr{p(x)}$ is a field.
\end{enumerate}
}{\\
Admitted
}


\end{document}
