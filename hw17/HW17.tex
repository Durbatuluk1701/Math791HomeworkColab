% This template retrieved from "https://guides.nyu.edu/LaTeX/templates"

\documentclass[11pt]{article} 

\usepackage{geometry} 
\usepackage{amsmath}  
\usepackage{graphicx} 
\usepackage{amssymb}
\usepackage{amscd}
\usepackage{amsfonts}
\usepackage[shortlabels]{enumitem}

\newcommand{\prob}[3]{\begin{flushleft}
        \textbf{Problem #1}: \\
        #2 
		\textbf{Solution:} 
		#3

\end{flushleft}}

\newcommand{\admit}{
  \begin{flushright}
    \textbf{Admitted}
  \end{flushright}
}

\newcommand{\makeHWtitle}[1]{
    \begin{center}
    \Large{Homework #1 - MATH 791} 
        \vspace{5pt}
        
        \normalsize{Will Thomas}
        \vspace{5pt}
    \end{center}
}

\begin{document}

\makeHWtitle{17}
Let $R$ be an integral domain. In what follows, $a,b,c,d,e,f \in R$ will be non-zero, non-unit elements.
Given $a,b \in R, d \in R$ is said to be a \emph{greatest common divisor}, or GCD, of $a$ and $b$ if the
following conditions hold:
\begin{enumerate}[(i)]
  \item $d \mid a$ and $d \mid b$
  \item Whenever $e \mid a$ and $e \mid b$, then $e \mid d$
\end{enumerate}
Use this definition to prove the following problems.

\prob{1}{
  Show that if GCDs exist, they are unique up to a unit multiple. \\
}{ \\
  Let us assume that we have two GCD's $d_1, d_2$ that are GCD's for $a$ and $b$.

  This means
  $$\exists q_1, q_1', q_2, q_2',\ a = q_1*d_1 \land a = q_2 * d_2 \land b = q_1' * d_1 \land b = q_2' * d_2$$

  Let us have $ab = q_1 * d_1 * q_2' * d_2$ and $ab = q_1' * d_1 * q_2 * d_2$, we can factor out and set these equal to get $q_1 * q_2' = q_1' * q_2$

  We also know by (ii) that $d_1 \mid d_2 \land d_2 \mid d_1$.
  This implies that $d_1 = u_1 d_2 \land d_2 = u_2 d_1 \implies d_1 = u_1 u_2 d_1 \implies 1 = u_1 u_2$

  Thus $u_1, u_2$ are units, which means that $d_1 = u_1 d_2 \implies $ $d_1$ and $d_2$ only differ by unit multiples.

  $$\therefore \text{GCD's are unique up to a unit multiple}$$
}

\prob{2}{
  Suppose $d_1$ is a GCD of $ab$ and $ac$, and $d_2$ is a GCD of $b$ and $c$.
  Prove that, $d_1$ is a unit multiple of $ad_2$.
  Use this to show that if $d$ is a GCD of $a$ and $b$, then $1$ is a GCD of $a/d$ and $b/d$ \\
}{ \\
  We know that $d_1 \mid ab \land d_1 \mid ac \land d_2 \mid b \land d_2 \mid c$.
  $$d_1 = q_1 ab \land d_1 = q_2 ac \land d_2 = q_3 b \land d_2 = q_4 c$$
  $$d_2 \mid b \implies d_2 \mid ab \land d_2 \mid c \implies d_2 \mid ac$$
  $$\implies d_2 \mid d_1$$
  So we can find $d_1 = q d_2$, $q_1 a b = q q_3 b \implies q_1 a = q q_3$ also $q_2 a c = q q_4 c \implies q_2 a = q q_4$
  $$q_1 a + q_2 a = q q_3 + q q_4 \implies (q_1 + q_2) a = (q_3 + q_4) q \implies (q_1 + q_2)/(q_3 + q_4) a = q$$
  $$d_1 = (q_1 + q_2)/(q_3 + q_4) a d_2$$
  We need to now show that $(q_1 + q_2)/(q_3 + q_4)$ is a unit.

  \admit

  We can use this to prove that $1 = GCD(a/d, b/c)$

  Taking $d_1 \mid a \land d_1 \mid b$, and then also $d_2 \mid ad_1' \land d_2 \mid bd_1'$, we know then that
  $d_2 = u_1 * a * d_1$, or also $d_2 = u_2 * b * d_1$.
  We can use this to reduce to $u_1 * a * d_1 \mid ad_1' \implies u_1 * d_1 \mid d_1' \land u_2 * b * d_1 \mid b d_1'$

  Since $u_1, u_2$ are units,

  \admit
}

\prob{2 (Solution 2)}{}
{
\begin{align*}
&d_2 \mid b, d_2 \mid c\\
&\Rightarrow d_2f_b = b, d_2f_c = c\\
&\Rightarrow ad_2f_b = ab, ad_2f_c = ac\\
&\Rightarrow ad_2 \mid ab, ad_2 \mid ac\\
&\Rightarrow ad_2 \mid d_1 \text{ from definition of gcd}\\
&\\
&\Rightarrow ad_2q = d_1\\
&\Rightarrow ad_2qf = d_1f = ab \text{ Since $d_1 \mid ab$}\\
&\Rightarrow ad_2qf = ab \Rightarrow d_2qf = b\\
&\text{Similarly,}\\
&\Rightarrow ad_2qh = d_1h = ac \text{ Since $d_1 \mid ac$}\\
&\Rightarrow ad_2qh = ac \Rightarrow d_2qh = c\\
&d_2q \mid c, d_2q \mid b \Rightarrow d_2q \mid d_2\\
&\Rightarrow \text{$q$ is a unit, so}\\
&ad_2q = d_1
\end{align*}
This means that $d_1$ is a unit multiple of $ad_2$.
}


\prob{3}{
  Show that if $1$ is a GCD of $a$ and $b$ is and $1$ is also a GCD of $a$ and $c$, then $1$ is a GCD of $a$ and $bc$. \\
}{ \\
  $\gcd(a,b) = 1 \implies 1 \mid a \land 1 \mid b$ and $\forall e, e \mid a \land e \mid b \implies e \mid d$.

  From this, we can conclude that the only possible values $e$ could take are $1$ as $\nexists e > 1$ s.t. $e \mid d$

  Similarly for $\gcd(a,c) = 1 \implies 1 \mid a \land 1 \mid b$ and $\forall e, e \mid a \land e \mid b \implies e \mid d$.

  This also shows that the only possible values for $e$ are $1$.

  To find $\gcd(a, bc)$ we need a value $d$ such that $d \mid a \land d \mid bc$.

  $$d \mid bc \iff d \mid b \lor d \mid c$$

  Combining this, we get that we need a value $d$ such that
  $$d \mid a \land (d \mid b \lor d \mid c) \iff (d \mid a \land d \mid b) \lor (d \mid a \land d \mid c)$$

  We proved earlier that for $(d \mid a \land d \mid b)$ the only values $d$ can take are $1$, and also $(d \mid a \land d \mid c)$ the only values $d$ can take are also $1$.

  $$\therefore \gcd(a,bc) = 1$$
}

\prob{4}{
  Show that if $R$ is a PID, and $a, b \in R$, then $d$ is a GCD of $a$ and $b$ if and only if $\langle a, b \rangle = \langle d \rangle$.
  In particular, every two non-zero, non-units have a GCD, and if $d$ is a GCD of $a$ and $b$, then $d = ra + sb$ for some $r, s \in R$ \\
}{ \\
  $$\langle d \rangle = r_1 d,\ \forall r_i \in R;\
    \langle a, b \rangle = r_a a + r_b b,\ \forall r_i \in R$$
  Proving $d = \gcd(a,b) \implies \langle a, b \rangle = \langle d \rangle$:

  If $d = \gcd(a,b) \implies d \mid a \land d \mid b$;
  we need to show $\forall r_a, r_b \in R,\exists r_d \in R,\ r_a a + r_b b = r_d d$
  $$r_a a + r_b b = r_d d = (r_a (d_a * d)) + (r_b (d_b * d)) = (r_d d)$$
  $$= (r_a d_a) d + (r_b d_b) d = (r_a d_a + r_b d_b) d = (r_d) d$$
  So if we pick $r_d := (r_a d_a + r_b d_b) \in R$ this holds

  Proving $\langle a, b \rangle = \langle d \rangle \implies d = \gcd(a,b) $:

  If $\langle a, b \rangle = \langle d \rangle \implies \forall r_a, r_b \in R, \exists r_d \in R, r_a a + r_b b = r_d d$; since
  $$d \mid (r_d d) \implies d \mid (r_a a + r_b b) \implies d \mid (r_a a) \land d \mid (r_b b)$$

  Since $R$ is a PID, that means that it is also a PID
  \admit
}

\prob{5}{
  Let $R = \mathbb{Q}[x,y]$ be the polynomial ring in two variables over $\mathbb{Q}$. Show that $1$ is a GCD of $x$ and $y$,
  but there is no equation of the form $1 = f \cdot x + g \cdot y$ for $f, g \in R$ \\
}{ \\
  We know fundamentally that $1 \mid x \land 1 \mid y$ as we can write $x = 1 * x \land y = 1 * x$.

  To show it is the \emph{greatest} common divisor, let us assume some divisor $\exists d \in R,$ s.t. $d > 1$.
  $$\implies d_x \mid x \land d_y \mid y \implies x = d_x * x' \land y = d_y * y' \land d_x = d_y$$

  We have two cases for each variable $x', y'$:

  Assume that $a,b$ are constants in $R$ and $f, f', g, g'$ are functions comprised solely of their respective variables.
  We also assume that any $f(x) + g(y)$ sum is not a constant (as it would fall into the constant case instead).

  \begin{center}
    \begin{tabular}{| c | c | c | c |} \hline
      $x'$          & $y'$            & $d_x$                   & $d_y$                     \\ \hline
      Constant $a$  & Constant $b$    & $\frac{x}{a}$           & $\frac{y}{b}$             \\ \hline
      Constant $a$  & $f'(x) + g'(y)$ & $\frac{x}{a}$           & $\frac{y}{f'(x) + g'(y)}$ \\ \hline
      $f(x) + g(y)$ & Constant $b$    & $\frac{x}{f(x) + g(y)}$ & $\frac{y}{a}$             \\ \hline
      $f(x) + g(y)$ & $f'(x) + g'(y)$ & $\frac{x}{f(x) + g(y)}$ & $\frac{y}{f'(x) + g'(y)}$ \\ \hline
    \end{tabular}
  \end{center}

  It is straightforward to see from this table that we can never reconcile $d_x = d_y$ unless we are in the case where $x' = f(x) + g(y) = x$ and $y' = f'(x) + g'(y) = y$ in which case
  $$d_x = \frac{x}{x} = 1 = \frac{y}{y} = d_y$$
  \textbf{Besides this}, these are just polynomials so we cannot have $d$ with $f(x) + g(y)$ in the denominator.

  $$\therefore \gcd(x, y) = 1$$

  Now, to show that no equation can be formed.

  Let us presume that $\exists f, g$ s.t. $1 = f x + g y$,
  however we need either $fx, gy$ or both to have a constant term.

  However, neither $fx$ or $gy$ can have a constant term, so
  $1 = fx + gy$ cannot have a constant term
  $$\therefore \nexists f, y \in R,\ s.t\ 1 = fx + gy$$
}



\end{document}
