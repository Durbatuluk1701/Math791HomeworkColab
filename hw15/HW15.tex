% This template retrieved from "https://guides.nyu.edu/LaTeX/templates"

\documentclass[11pt]{article} 

\usepackage{geometry} 
\usepackage{amsmath}  
\usepackage{graphicx} 
\usepackage{amssymb}
\usepackage{amscd}
\usepackage{amsfonts}
\usepackage[shortlabels]{enumitem}

\newcommand{\prob}[3]{\begin{flushleft}
        \textbf{Problem #1}: \\
        #2 
		\textbf{Solution:} 
		#3

\end{flushleft}}

\newcommand{\makeHWtitle}[1]{
    \begin{center}
    \Large{Homework #1 - MATH 791} 
        \vspace{5pt}
        
        \normalsize{Will Thomas}
        \vspace{5pt}
    \end{center}
}

\begin{document}

\makeHWtitle{15}

In this assignment, you will verify that the ring $R = \mathbb{Z}[\sqrt{-5}] := \{ a + b\sqrt{-5} \mid a, b \in \mathbb{Z} \}$
does not have the unique factorization property. The \emph{norm} from $R$ is $\mathbb{Z}$ is defined as follows: For $x = a + b\sqrt{-5}$,
$N(x) := a^2 + 5b^2$

\prob{1}{
    Show that $N(xy) = N(x) N(y)$ for all $x, y \in R$ \\
}{ \\
    For any $x, y \in R, \exists a_x b_x a_y b_y \in \mathbb{Z},\ x = a_x + b_x\sqrt{-5} \land y = a_y + b_y\sqrt{-5}$

    $$N(xy) = N(a_x + b_x\sqrt{-5} \cdot a_y + b_y\sqrt{-5}) = N(a_x + b_x\sqrt{-5} \cdot a_y + b_y\sqrt{-5})$$
    $$= N(a_x a_y + a_x b_y\sqrt{-5} + b_x\sqrt{-5}a_y + b_x\sqrt{-5} b_y \sqrt{-5})$$
    Since $\mathbb{Z}$ is a commutative ring, we can assume that $R$ is which will let us reduce to
    $$= N((a_x a_y - 5b_xb_y) + (a_x b_y + a_y b_x)\sqrt{-5})$$
    $$= (a_x a_y - 5b_xb_y)^2 + 5((a_x b_y + a_y b_x))^2$$
    $$= (a_x a_y - 5b_xb_y)(a_x a_y - 5b_xb_y) + 5((a_x b_y + a_y b_x)(a_x b_y + a_y b_x))$$
    $$= a_x^2 a_y^2 + 25 b_x^2 b_y^2 - 10 a_x a_y b_x b_y + 5(a_x^2 b_y^2 + 2 a_x a_y b_x b_y + a_y^2 b_x^2)$$
    $$= a_x^2 a_y^2 + 5a_x^2 b_y^2 + 5a_y^2 b_x^2 + 25 b_x^2 b_y^2$$
    $$= (a_x^2 + 5 b_x^2)(a_y^2 + 5 b_y^2) = N(x) N(y)$$
    $$\therefore \forall x, y \in R,\ N(xy) = N(x)N(y)$$
}

\prob{2}{
    Use the norm to describe the units in $R$. \\
}{ \\
    We said that a unit was an element $u \in R$ such that $\exists u^{-1} \in R$ and $uu^{-1} = 1 = u^{-1}u$

    We know from problem one that $N(1) = N(uu^{-1}) = N(u)*N(u^{-1})$ and $N(1) = 1^2 + 0 = 1$

    Since we are in the integers, we can then conclude that the only way an element can be a unit is if $N(x) = 1$, as $N(1) = 1 \implies 1 = N(u)*N(u^{-1})$ and we cannot have fractions in $\mathbb{Z}$
    $$\therefore \forall x \in R, \text{$x$ is a unit $\iff N(x) = 1$}$$
}

\prob{3}{
    Show that $3,2+ \sqrt{-5}, 2 - \sqrt{-5}$ are irreducible elements in $R$. \\
}{ \\
    An element $x$ is "irreducible" if whenever $q = ab$ for $a, b \in R$ that either $a$ or $b$ is a unit.

    To show that those elements are irreducible, we need to show that any way to factor them must involve a unit

    Let $3 = a*b$, we also know that $N(3) = N(a)*N(b) \implies N(3) = 3^2 = 9 = N(a)*N(b)$
    This is also equivalent to saying that $N(a) \mid 9 \land N(b) \mid 9$ (in the integers not in $R$)
    The factors of $9$ in $\mathbb{Z}$ are $\{ 1, 3, 9 \}$, in the cases where we have $N(a \text{ or } b) = 1, 9$ then we will
    have solved that one of the factors is a unit as one of them must have $N(a \text{ or } b) = 1$ which means its a unit by problem 2.

    If the factors are $N(a) = 3 \land N(b) = 3$, then we will arrive at a contradiction. Let us assume $a := x_a + y_a\sqrt{-5}$
    $\implies N(a) = N(x_a + y_a\sqrt{-5}) = x_a^2 + 5y_a^2 = 3$
    $\implies 3 - 5y_a y_a = x_a x_a \implies x_a \mid (3 - 5 y_a y_a) \implies x_a \mid 3$
    However, $3$ is prime in $\mathbb{Z} \implies x_a = \{ 1, 3 \}$.
    Similarly, $3 - x_a x_a = 5 y_a y_a \implies 3 - 1 = 2 = 5 y_a y_a$ which cannot work for any $y_a \in \mathbb{Z}$.
    For $3 - 9 = -6 = 5 y_a y_a$ this also cannot work for any $y_a \in \mathbb{Z}$
    $$\therefore \text{$3$ is irreducible in $R$}$$

    Very similar arguments can be used for the other two elements. All elements have $N(x) = 9$ and the remaining argument is based upon $9$ not the original element.
}

\prob{4}{
    Use the equation $3 \cdot 3 = (2 + \sqrt{-5}) \cdot (2 - \sqrt{-5})$ to show that $3, 2 + \sqrt{-5}, 2 - \sqrt{-5}$ are not prime in $R$.

    Conclude that $R$ does not have the unique factorization property. \\
}{ \\
    We call a number "prime" if when $p \mid ab \implies p \mid a \lor p \mid b$

    Using $3 \cdot 3 = (2 + \sqrt{-5}) \cdot (2 - \sqrt{-5})$, where they are all irreducible, this means that
    $$3 \mid (2 + \sqrt{-5}) \cdot (2 - \sqrt{-5}) \implies 3 \mid (2 + \sqrt{-5}) \lor 3 \mid (2 - \sqrt{-5})$$
    $$\implies (2 \pm \sqrt{5}i) = 3(a \pm b\sqrt{5}i)$$
    $$\implies 2 = 3a \rightarrow\leftarrow$$
    As $2 \neq 3a \in \mathbb{Z}$
    $$\therefore \text{$3, 2 + \sqrt{-5}, 2 - \sqrt{-5}$ are not prime in $R$}$$

    We can conclude from this that $R$ does not have the unique factorization property as
    $9 = 3 \cdot 3 = (2 + \sqrt{-5}) \cdot (2 - \sqrt{-5})$ cannot be factored \textbf{uniquely} in $R$
}

\prob{5}{
    Show that the ideal of $R$ generated by $3$ and $2 + \sqrt{-5}$ is not a \emph{principal ideal}, i.e.,
    there does not exist $f \in R$ such that $\langle 3, 2 + \sqrt{-5} \rangle = \langle f \rangle$ \\
}{ \\
    Let us assume $\exists f \in R$ such that $\langle 3, 2 + \sqrt{-5} \rangle = \langle f \rangle$

    In order for this to occur we need an $f$ such that $\exists n, f^n = 2 + \sqrt{-5}$ and $\exists m, f^m = 3$.
    However, $2 + \sqrt{-5}$ is irreducible which means that $f = 2 + \sqrt{-5}$ by the first equation.

    We also know that $\exists m, (2 + \sqrt{-5})^m = 3$ can never work as $3$ is irreducible and $(2 + \sqrt{-5})$ is not a unit.
}



\end{document}
