% This template retrieved from "https://guides.nyu.edu/LaTeX/templates"

\documentclass[11pt]{article} 

\usepackage{geometry} 
\usepackage{amsmath}  
\usepackage{graphicx} 
\usepackage{amssymb}
\usepackage{amscd}
\usepackage{amsfonts}
\usepackage[shortlabels]{enumitem}

\newcommand{\prob}[3]{\begin{flushleft}
        \textbf{Problem #1}: \\
        #2 
		\textbf{Solution:} 
		#3

\end{flushleft}}

\newcommand{\admit}{
  \begin{flushright}
    \textbf{Admitted}
  \end{flushright}
}

\newcommand{\lr}[1]{
  \langle #1 \rangle
}

\newcommand{\makeHWtitle}[1]{
    \begin{center}
    \Large{Homework #1 - MATH 791} 
        \vspace{5pt}
        
        \normalsize{Will Thomas}
        \vspace{5pt}
    \end{center}
}

\begin{document}

\makeHWtitle{25}

\prob{1}{
  Consider $\alpha := 1 + \sqrt{2} + \sqrt{3} + \sqrt{6} \in \mathbb{Q}(\sqrt{2}, \sqrt{3})$. 
  Find a polynomial $p(x) \in \mathbb{Q}[x]$ such that $p(\alpha) = 0$. Determine if the polynomial you found is the minimal polynomial for $\alpha$ over $\mathbb{Q}$. \\
}{\\
\admit
}

\prob{2}{
  Prove that for any field $L$ containing $\mathbb{Z}_p$ and $a, b \in L$, then $(a + b)^p = a^p + b^p$. \\
}{\\
\admit
}

\prob{3}{
  Let $x,y$ be indeterminates over the field $\mathbb{Z}_2$. Set $F := \mathbb{Z}_2(x^2, y^2)$ and $K := \mathbb{Z}_2(x,y)$. Set $E := \mathbb{Z}(x,y^2)$.
  Prove that $[E : F] = 2$ and $[K : E] = 2$.
  
  Conclude that $[K : F] = 4$. \\
}{\\
\admit
}

\prob{4}{
  In the notation of problem 3, prove that $\alpha^2 \in F$, for all $\alpha \in K$. \\
}{\\
\admit
}

\prob{5}{
  Use the previous two problems to show that for $F \subseteq K$ as in problem 3, there does \textbf{not} exist $\alpha \in K$ such that $K = F(\alpha)$.
  Conclude that there are infinitely many intermediate field $F \subsetneq E \subsetneq K$. \\
}{\\
\admit
}

\end{document}
