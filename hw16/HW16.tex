% This template retrieved from "https://guides.nyu.edu/LaTeX/templates"

\documentclass[11pt]{article} 

\usepackage{geometry} 
\usepackage{amsmath}  
\usepackage{graphicx} 
\usepackage{amssymb}
\usepackage{amscd}
\usepackage{amsfonts}
\usepackage[shortlabels]{enumitem}

\newcommand{\prob}[3]{\begin{flushleft}
        \textbf{Problem #1}: \\
        #2 
		\textbf{Solution:} 
		#3

\end{flushleft}}

\newcommand{\makeHWtitle}[1]{
    \begin{center}
    \Large{Homework #1 - MATH 791} 
        \vspace{5pt}
        
        \normalsize{Will Thomas}
        \vspace{5pt}
    \end{center}
}

\begin{document}

\makeHWtitle{16}
Throughout this assignment, $R$ is an integral domain. The first three problems show that we can construct a field
containing $R$ in the exact manner that the rational numbers are constructed from the integers.
Recall, that formally speaking, the rational numbers are the set of equivalence classes of ordered pairs $(a,b)$ of integers
(with $b \neq 0$) such that $(a,b)$ is equivalent to $(c,d)$ if and only if $ad = bc$. Of course,
we denote the equivalence class of an ordered pair $(a,b)$ as $a/b$

\prob{1}{
  Let $Q$ denote the set of ordered pairs $(a,b)$ with $a,b \in R$ and $b \neq 0$. For $(a,b), (c,d) \in Q$, define $(a,b) \sim (c,d) \iff ad = bc \in R$.
  Show that $\sim$ is an equivalence relation. \\
}{ \\
  To show equivalence we need:

  Reflexive:
  $$(a,b) \sim (a,b) \forall a, b \in R$$
  $$(a,b) \sim (a,b) \iff ab = ba$$
  We know that since all ID's are commutative rings that this must hold.

  Symmetric:
  $$\forall a, b \in R, (a,b) \sim (b,a)$$
  $$(a,b) \sim (b,a) \iff ab = ab$$
  Trivially holds

  Transitivity:
  $$(a,b) \sim (c,d) \land (c,d) \sim (e,f) \implies (a,b) \sim (e,f)$$
  $$(a,b) \sim (c,d) \implies ad = bc,\ (c,d) \sim (e,f) \implies cf = de$$
  Multiplying both sides by $ef$ we get $adef = bcef$ which using that fact that this is an ID, we can use commutativity and
  cancellation to reach
  $$af(de) = be(cf) \implies af = be \iff (a,b) \sim (e,f)$$

  $$\therefore \text{This is an equivalence relation}$$
}

\prob{2}{
Let $K$ denote the set of equivalence classes under the equivalence relation in 1. Temporarily using $[(a,b)]$ to denote the equivalence class of $(a,b)$,
defined addition and multiplication of elements in $K$ as follows:
$$[(a,b)] + [(c,d)] := [(ad + bc, bd)]; \ [(a,b)] \cdot [(c,d)] = [(ac, bd)]$$
Show that addition and multiplication in $K$ are well defined. \\
}{ \\
Addition:

To show well-defined, let us take $(a,b) \sim (c,d) \in K$ and $(e,f) \in K$ and show that
$$[(a,b)] + [(e,f)] \sim [(c,d)] + [(e,f)]$$
$$[(a,b)] + [(e,f)] = [(af + be, bf)],\ [(c,d)] + [(e,f)] = [(cf + de, df)]$$
We want to show that $[(af + be, bf)] \sim [(cf + de, df)]$
$$[(af + be, bf)] \sim [(cf + de, df)] \iff adf^2 + bdef = bcf^2 + bdef$$
We can cancel out the $bdef$ and then apply the fact that $(a,b) \sim (c,d) \implies ad = bc$ to solve this problem.
$$\therefore \text{Addition is well-defined}$$

Multiplication:

To show well-defined, let us take $(a,b) \sim (c,d) \in K$ and $(e,f) \in K$ and show that
$$[(a,b)] \cdot [(e,f)] \sim [(c,d)] \cdot [(e,f)]$$
$$[(a,b)] \cdot [(e,f)] = [(ae, bf)],\ [(c,d)] \cdot [(e,f)] = [(ce, df)]$$
We want to show that $[(ae, bf)] \sim [(ce, df)]$
$$[(ae, bf)] \sim [(ce, df)] \iff adef = bcef$$
We can cancel out the $ef$ and then apply the fact that $(a,b) \sim (c,d) \implies ad = bc$ to solve this problem.
$$\therefore \text{Multiplication is well-defined}$$
}

\prob{3}{
Show that $K$ is a field under the operations above and that the set of elements in $K$ of the form $[(a,1)]$ is a subring of $K$ isomorphic to $R$.
The field $K$ is called the \emph{quotient field} of $R$ or \emph{fraction field} of $R$. \\
}{ \\
To show $K$ is a field, we need to show that $+$, $\cdot$ and commutative, associative, have identities and inverses, and that $\cdot$ distributes over $+$.

The commutativity and associativity are fairly obvious from the definition and the fact that the underlying ring $R$ is a ID.

The additive identity will be $[(0,1)] + [(a,b)] = [(a, b)]$ and the multiplicative identity will be $[(1,1)] \cdot [(a,b)] = [(a,b)]$

The additive inverses will be $[(a,b)] + [(a, -b)] = [(-ab + ab = 0,1)]$,
the multiplicative inverse will be $[(a,b)] \cdot [(b,a)] = [(ab, ab)] \sim [(1,1)]$ as $[(ab, ab)] \sim [(1,1)] \iff ab = ab$ which is obviously true, thus $[(ab,ab)] \in [(1,1)]$ so it is the identity.

For distributivity, $[(a,b)] \cdot ([(c,d)] + [(e,f)]) = [(acf + ade, bdf)] = [((ad + bc), bd)] + [((af + be), bf)]$
$= [(a,b)] \cdot [(c,d)] + [(a,b)] \cdot [(e,f)]$

To show that the subring is isomorphic, we can use the First Isomorphism theorem and define $\phi : K \rightarrow R$ by $\phi([(a,b)]) = a$.
$$\ker(\phi) = \{ [(a,b)] \in K \mid \phi([(a,b)]) = 0 \}$$
$$\ker(\phi) = \{ [(0,b)] \in K \}$$

To show this is a ring homomorphism
$$\forall [(a,b)], [(c,d)] \in K, \phi([(a,b)] [(c,d)]) = \phi([(a,b)]) \phi([(c,d)])$$
$$\phi([(a,b)] [(c,d)]) = \phi([ac, bd]) = ac = \phi([(a,b)]) \phi([(c,d)])$$
Similarly for addition
$$\forall [(a,b)], [(c,d)] \in K, \phi([(a,b)] + [(c,d)]) = \phi([(a,b)]) + \phi([(c,d)])$$
$$\phi([(a,b)] + [(c,d)]) = \phi([(ad + bc, bd)]) = ad + bc = a + c$$
And since we are only taking elements such that the second element is $1$, that means $a1 + 1c = a + c = \phi([(a,b)]) + \phi([(c,d)])$

It is fairly straightforward that $\phi$ is surjective

$$\therefore \text{The subring of $K$ formed by $[(a,1)]$ is isomorphic to $R$}$$
}
\textbf{Remark:} \\
Henceforth we will write the elements of $K$ as $a/b$, rather than $[(a,b)]$ and an element $a \in R$ either as $a$ or $a/1$ and regard $R$
as a subring of $K$. Note then that $a/b + c/d = (ad + bc)/bd$ and $a/b \cdot c/d = ac/bd$ as expected

\prob{4}{
  Let $L$ be a field containing $R$. Show that $L$ contains $K$ (or at least an isomorphic copy of $K$). Thus in this sense, $K$ is
  the smallest field containing $R$. \\
}{ \\
  If $L$ is a field containing $R$, let us assume for contradiction that $\exists k_1/k_2 \in K$ that is not in $L$

  We have two cases, either $k_1/k_2 \in R$ which means it is in $L \rightarrow\leftarrow$

  So we must be in the case where $k_1/k_2 \notin R \iff k_2 \neq 1$. If $k_1/k_2 \in K \iff k_2/k_1 \in K$ also.
  Yet by a similar argument as above, $k_2/k_1 \notin R \iff k_1 \neq 1$

  Additionally, since $K$ is made of equivalence classes, we know that $k_1 \nmid  k_2 \land k_2 \nmid k_1$.
  This is to say that $k_1$ and $k_2$ are relatively prime.

  Since we proved the Isomorphism earlier, we know that $k_1/1 \in R \land k_2/1 \in R$, and since $L$ is a \emph{field} containing $R$, that means it must have multiplicative inverses $\implies 1/k_1 \in L \land 1/k_2 \in L$. We can then show that $k_1/1 \cdot 1/k_2 = k_1/k_2 \in L$

  $$\therefore \text{No element in $K$ cannot be in $L$}$$
  $$\therefore \text{$K$ is the smallest field containing $R$}$$
}

\prob{5}{
Let $A$ be an $m \times n$ matrix with entries in $R$ satisfying $m < n$. Set $x := \begin{pmatrix}
    x_1 \\ \vdots \\ x_n
  \end{pmatrix}$ and $0 := \begin{pmatrix}
    0 \\ \vdots \\ 0
  \end{pmatrix}$.
Use standard facts from linear algebra to show that the homogeneous system of equations $A \cdot x = 0$ has infinitely many
solutions over $R$ \\
}{ \\
First let us enumerate $A$ as $A := (a_{ij})$ for $i < m$ and $n < j$

$$A \cdot x = \begin{pmatrix}
    a_{11} x_1 + \cdots + a_{1n} x_n \\
    \vdots                           \\
    a_{m1} x_1 + \cdots + a_{mn} x_n
  \end{pmatrix} = 0$$

% We can reduce this matrix as there must be a minimal $a_{ij}$ that can be factored out of the rest of the $a_{i'j'}$ in each column.
% This means that we can wipe out all but a single row and just be left with $a_{ij} x_1 + \cdots + a_{ij} x_n = 0$

Let us start by just finding the solutions for the top row, $a_{11} x_1 + \cdots + a_{1n} x_n = 0$.
If $n = 1$ then that means that $m < n \implies m = 0$ so that is ill-posed.

Instead $n \geq 2$, so $a_{11} x_1 + a_{12} x_2 = 0$ then we know that $a_{11} x_1 = - a_{12} x_2$.
We can then claim WLOG that $a_{12} = a' \cdot x_1 \cdot x_2^{-1}$
$$a_{11} x_1 = - a' x_1 \implies a_{11} = -a'$$
Since $a_{11}$ is left arbitrary, this means that it can be any element in $R$, and $R$ is infinite
$$\therefore \text{$A \cdot x = 0$ has infinitely many solutions}$$

}

\end{document}
